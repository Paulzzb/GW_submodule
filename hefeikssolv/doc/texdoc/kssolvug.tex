\documentclass[11pt]{book}
\setlength{\topmargin}{-.5in} 
\setlength{\textheight}{8.8in}
\setlength{\textwidth}{6.2in} 
\setlength{\oddsidemargin}{0.2in}
\oddsidemargin-.0in
\evensidemargin-.0in
\usepackage{amsmath,amssymb,amsfonts,color}
\usepackage{epsfig, moreverb, graphicx}
\usepackage[TABTOPCAP,tight]{subfigure}
\usepackage{epsfig}


\newtheorem{proposition}{Proposition}
\newtheorem{algorithm}{Algorithm}
\newtheorem{remark}{{\it Remark}}
\newtheorem{theorem}{Theorem}
\newtheorem{lemma}{Lemma}
\newtheorem{corollary}{Corollary}

\newcommand{\xij}{x_j^{(i)}}
\newcommand{\pij}{p_j^{(i)}}
\newcommand{\exz} {X^{(0)}}
\newcommand{\exi} {X^{(i)}}
\newcommand{\exim} {X^{(i-1)}}
\newcommand{\exip} {X^{(i+1)}}
\newcommand{\Hz} {H(X^{(0)})}
\newcommand{\Hi} {H^{(i)}}
\newcommand{\Him} {H^{(i-1)}}
\newcommand{\ri} {R^{(i)}}
\newcommand{\pim} {P^{(i-1)}}
\newcommand{\Ehat}{\hat{E}}
\newcommand{\XS}{X^{(0)}}
\newcommand{\hEtot}{\hat{E}_{total}}
\newcommand{\rhoi}{\rho^{(i)}}

\newcommand{\rhok}{\rho^{(k)}}
\newcommand{\muk}{\mu^{(k)}}
\newcommand{\exk} {X^{(k)}}
\newcommand{\exkm} {X^{(k-1)}}
\newcommand{\exkp} {X^{(k+1)}}
\newcommand{\rhostar} {\rho_{\tiny \ast}}
\newcommand{\rhoopt} {\rho_{\tiny \mbox{opt}}}

\newcommand{\tK}{\widetilde{K}}
\newcommand{\tM}{\widetilde{M}}
\newcommand{\hK}{\widehat{K}}
\newcommand{\hM}{\widehat{M}}
\newcommand{\hG}{\widehat{G}}
\newcommand{\hV}{\widehat{V}}
\newcommand{\hx}{\widehat{x}}
\newcommand{\hy}{\widehat{y}}
\newcommand{\bE}{{\bf E}}
\newcommand{\bn}{{\bf n}}
\newcommand{\Rnk}{\mathbb{R}^{n\times k}}
\newcommand{\Rnn}{\mathbb{R}^{n\times n}}
\newcommand{\Rkk}{\mathbb{R}^{k\times k}}
\newcommand{\rhoin}{\rho_{\mbox{in}}}
\newcommand{\rhout}{\rho_{\mbox{out}}}

\newcommand{\qed}{\hfill \ensuremath{\Box}}
\def\sss{\scriptstyle}

\newcommand{\ignore}[1]{}

\begin{document}
\title{KSSOLV User's Guide}
\author{
Chao Yang\\
Juan C. Meza\\
Computational Research Division\\
Lawrence Berkeley National Laboratory \\
Berkeley, CA 94720
}
\date{\today}
\maketitle
\tableofcontents

\chapter{Introduction} \label{chap:intro}
KSSOLV is a MATLAB toolbox for solving a class of nonlinear eigenvalue
problems also known as the Kohn-Sham equations. This type of problem arises 
from electronic structure calculation which is nowadays an essential tool 
for studying the microscopic quantum mechanical properties of molecules,
solids and other nanoscale materials.  Through the density functional theory (DFT) 
formalism, one can reduce the many-body Schr\"{o}dinger equation used to describe the 
electron-electron and electron-nuclei interactions within an atomistic
system to a set of single-electron equations that have far
fewer degrees of freedom.  These equations are first derived by Kohn and Sham
in \cite{kohnsham}, and they have the form
\begin{equation}
H(\rho) \psi_i = \lambda_i \psi_i, \ \ i = 1, 2, ..., n_e, \label{ev1} 
\end{equation}
where $n_e$ is the number of electrons in the system, 
$\{\psi_i(r)\}$ ($i = 1,2,..., n_e$) is a set of single electron
wavefunctions that are mutually orthonormal, $\rho(r)$ is the total electron
charge defined by
\begin{equation}
\rho(r) =  \sum_{i=1}^{n_e} | \psi_i(r) |^2,
\label{chargeden}
\end{equation}
and 
\[
\lambda_1 \leq \lambda_2 \leq \hdots \leq \lambda_{n_e}
\] 
are the $n_e$ smallest eigenvalues of $H(\rho)$, a linear operator 
that depends on $\rho$.  The operator $H$ is known as the Kohn-Sham 
Hamiltonian and defined by
\begin{equation}
H(\rho) = -\frac{1}{2}\Delta + V_{ion}(r) + \rho \star \frac{1}{|r|}
+ V_{xc}(\rho),
\label{ksham}
\end{equation}
where $\Delta$ is the Laplacian operator, $V_{ion}$ and $V_{xc}$ are known as
the ionic and exchange-correlation potentials respectively \cite{kohnsham},
and $\star$ denotes the convolution operator.

The Kohn-Sham equations are the the Euler-Lagrange equations for
the total energy minimization problem
\begin{eqnarray}
\begin{array}{cc}
\min       & E_{total}^{KS}(\psi_i) \\
\mbox{s.t} & \psi_i^{\ast}\psi_j = \delta_{i,j},
\end{array} \label{ksmin}
\end{eqnarray}
where
\begin{equation}
E_{total}^{KS} = \frac{1}{2}\sum_{i=1}^{n_e} \int_{\Omega} 
|\nabla \psi_i|^2 dr + \int_{\Omega}\rho  V_{ion} dr
+\frac{1}{2}\int_{\Omega} \int_{\Omega} \frac{\rho(r_1)\rho(r_2)}{|r_1 - r_2|}
dr_1 dr_2 + E_{xc}(\rho),
\label{ksetot}
\end{equation}
$E_{xc}(\rho)$ is the exchange-correlation energy defined in \cite{kohnsham}. 

After (\ref{ev1}) is discretized, it becomes a set of nonlinear equations 
that bear similarity to algebraic eigenvalue problems found in standard linear 
algebra textbook. The main feature that distinguishes the Kohn-Sham equations
from the standard linear eigenvalue problem is that the matrix operators 
in these equations are functions of the eigenvectors to be computed.  
Due to this nonlinear coupling between the matrix operator and
its eigenvector, the Kohn-Sham equations are much more difficult to
solve than a standard linear eigenvalue problem.  
Currently, the most widely used numerical method for solving the 
this type of problem is the Self Consistent Field (SCF) iteration, 
which we will examine in detail in Chapter~\ref{chap:methods}. 
An alternative approach which is designed to minimize the 
total energy (\ref{ksetot}) directly will also be reviewed there.
Both methods have been implemented in KSSOLV to allow users to 
compute the solution to the Kohn-Sham equations associated with various
molecules and solids.

KSSOLV is designed using the object oriented programming features
available in MATLAB.  Several classes are created in KSSOLV to 
define both physical objects such as atoms and molecules and 
mathematical entities such as wavefunctions and Hamiltonians.
These classes are defined in a hierarchical fashion so that
a molecule object can be easily constructed from atom objects.
The use of objects not only allows users with a minimal physics or 
chemistry background to quickly assemble a realistic atomistic
system, but also allows them to develop and compare new numerical 
methods for solving the Kohn-Sham equations in a user friendly
environment. 

This user's guide will help users get familiar with various
features of the toolbox.  It is organized as follows. 
In next chapter, we will demonstrate how the toolbox
can be used to set up simple molecular systems and how the
Kohn-Sham equations associated with these systems can be solved
by calling the numerical methods currently implemented in KSSOLV.
In Chapter~\ref{chap:class}, we will take a closer look at
various classes defined in KSSOLV, and show how KSSOLV
objects can be created and modified.
Chapter~\ref{chap:methods} focuses on numerical methods for
solving the Kohn-Sham equations that have been  implemented 
in KSSOLV. In particular, we will examine the SCF iteration and
the DCM algorithm.  We will discuss the importance of using
the trust region technique to maintain monotonic reduction of the
total energy function and how charge mixing can be used to
accelerate and stabilize the convergence of the SCF iteration.
%The performance of the KSSOLV will be briefly discussed in 
%Chapter~\ref{chap:perf}.

\chapter{Getting Started} \label{chap:start}
%
\section{Download KSSOLV}
KSSOLV can be downloaded from 
\begin{verbatim}
    http://www.crd.lbl.gov/~chao/KSSOLV
\end{verbatim}

\section{Setting the environmental variable {\tt KSSOLVPATH}}
Once you retrieve the package, take a look at the README file in the directory
{\tt kssolv}. The file contains a minimal set of instructions for 
using the package.

Before running KSSOLV, one should define the environmental variable {\tt
  KSSOLVPATH}.  If you are running MATLAB under Unix/Linux operating system, this can
  be done by adding, for example, 
\begin{verbatim}
   setenv KSSOLVPATH  /home/cyang/kssolv
\end{verbatim}
to your .cshrc if you use {\tt csh}.  If you use {\tt bash}, use
\begin{verbatim}
   export KSSOLVPATH=/home/cyang/kssolv
\end{verbatim}

If you are running MATLAB under Windows, click on {\tt Start->My Computer}, 
and then click {\tt View system information} on the left ({\tt System Task}) 
panel. Continue with a click on {\tt Advanced}, followed by a click on 
{\tt Environmental variables}. If 
{\tt KSSOLVPATH} is not among the defined environmental variables, click new, 
type in the variable name KSSOLVPATH, and the directory in which KSSOLV is installed.

\section{Test Problems}
The package contains a number of test problems that one can experiment
with.  Each test problem represents a particular molecule or
periodic system. The system is created in a setup file.  Table~\ref{egtab}
contains the names of all setup files and a brief description for each
of them. It also shows the number of electron pairs (occupied states ({\tt nocc})
for most systems (with the exception of the electron quantum dot example in 
which {\tt nocc} is the actual number of electrons in the dot).  
%
\begin{table}[htbp]
\center
\begin{tabular}{|c|c|l|} \hline
 setup file name      &   nocc &  Description \\ \hline
{\tt c2h6\_setup.m}   &     7  &  an ethane molecule \\
{\tt co2\_setup.m}    &     8  &  a carbon dioxide molecule\\
{\tt h2o\_setup.m}    &     4  &  a water molecule \\
{\tt hnco\_setup.m}   &     8  &  an isocyanic acid molecule \\
{\tt qdot\_setup.m}   &     8  &  a 8-electron quantum dot confined 
                                  by an external potential\\
{\tt si2h4\_setup.m}  &     6  &  a planar singlet silylene molecule \\
{\tt sibulk\_setup.m} &     4  &  a silicon bulk system \\
{\tt sih4\_setup.m}   &     4  &  a silane molecule \\
{\tt ptnio\_setup.m}  &     43 &  a $Pt_2Ni_6O$ molecule \\
\hline
\end{tabular}
\caption{Setup files for examples included in KSSOLV}
\label{egtab}
\end{table}
%
\section{Setting up an Molecular System}
To create a new system, a user can simply take one of the existing setup 
files and modify the construction of the {\tt Molecule} object. A user can 
also change of the size of the primary cell or the kinetic energy cutoff of 
the existing setup file to examine the change in the convergence behavior 
of the numerical method or the quality of the computed solution.

In the following, we will use {\tt sih4\_setup.m} as an example to demonstrate
how a molecular system can be easily set up in KSSOLV.  The source code
contained in {\tt sih4\_setup.m} is shown in Figure~\ref{sih4source}.

\subsection{Initialization}
In KSSOLV, an molecular system is defined by first creating an {\tt Molecule}
object, an particular instance of the {\tt Molecule} class to be defined in
Chapter~\ref{chap:class}, using
\begin{verbatim}
   mol = Molecule();
\end{verbatim}
Here {\tt mol} is a user defined variable name and {\tt Molecule} is a KSSOLV
keyword. This single line of code creates an empty {\tt Molecule} object
{\tt mol} with no particular attribute.

\subsection{Specifying the Atom List}  
Properties of an {\tt Molecular} object are specified by using the {\tt set}
method associated with the object. For example,
\begin{verbatim}
   mol = set(mol,'atomlist', alist); 
\end{verbatim}
is used in Figure~\ref{sih4source} to specify the list of atoms contained in
{\tt mol}.  Here the string {\tt 'atomlist'} is the name of an attribute
associated with an {\tt Molecule} object predefined by KSSOLV, and {\tt alist}
is a user defined variable which supplies the actual atom list.  In
Figure~\ref{sih4source}, the variable {\tt alist} is defined earlier as
an array of {\tt Atom} objects. Each {\tt Atom} object can be defined by 
its chemical symbol or atomic number. For example, the silicon atom can 
be defined by
\begin{verbatim}
  a = Atom('Si')
\end{verbatim}
or
\begin{verbatim}
  a = Atom(14); 
\end{verbatim}
Here, {\tt Atom} is a KSSOLV keyword associated with the atom class 
(see Chapter~\ref{chap:class} for details.)

The atom list array {\tt alist} can also be defined by
\begin{verbatim}
    alist(1) = Atom('Si');
    for j = 2:5
       alist(j) = Atom('H');
    end
\end{verbatim}

\subsection{Specifying the Positions of the Atoms}  
The position of each atom in the atom list is specified by an $n_a \times 3$
array ({\tt xyzmat} in Figure~\ref{sih4source}). The $i$th row of this array
gives the $x,y,z$ coordinates of the $i$th atom in the atom list.  
This array can be passed to the {\tt mol} object through the {\tt set} 
method as one can see in Figure~\ref{sih4source}.  The unit of the coordinates
should be in Bohr (1 Bohr = 0.5291772083 Angstrom).

Although it is perfectly OK to define this array directly, it is sometimes
more convenient to first define the relative position of each atom
in a unit (super)cell (of a periodic lattice) specified by three lattice 
vectors ${\bf c}_1$, ${\bf c}_2$ and ${\bf c}_3$. If we denote the relative 
displacement of the atom along ${\bf c}_i$ by $\xi_i$, then the 
absolute $x,y,z$ coordinates of the atom correspond to the
three components of the row vector
\[
r = \xi_1 {\bf c}_1^T +  \xi_2 {\bf c}_2^T +  \xi_3 {\bf c}_3^T
  = (\xi_1 \: \xi_2 \: \xi_3) C,
\]
where
\begin{equation}
C = \left(
\begin{array}{c}
{\bf c}_1^T \\
{\bf c}_2^T \\
{\bf c}_3^T \\
\end{array}
\right).
\label{unitcel}
\end{equation}
The $3 \times 3$ matrix $C$ defines a unit supercell in KSSOLV. It can
be used to set the {\tt 'supercell'} attribute of a {\tt Molecule}
object.  For example, 
\begin{verbatim}
mol = set(mol,'supercell',10*eye(3));
\end{verbatim}
sets the unit supercell of {\tt mol} to a $10 \times 10 \times 10$ cube. 

If the relative positions of $n_a$ atoms within a unit supercell
defined by $C$ are placed in an $n_a \times 3$ array {\tt coefs}, 
then the absolute coordinates of these atoms can be obtained from
{\tt coefs*C}.

We should point out that the $(x,y,z)$ coordinates associated with
the atoms contained in a {\tt Molecule} object can be negative even 
though the lattice vectors used to specify the unit supercell are usually in 
$\mathbb{R}^+\times \mathbb{R}^+ \times \mathbb{R}^+$.  Because 
the unit supercell is assumed to periodically extended throughout
the entire domain, $(x,y,z)$ is equivalent to 
$(x+m{\bf c}_1,y+n{\bf c}_2, z+\ell{\bf c}_3)$ for any integer
$m$, $n$ and $\ell$, i.e., we can always bring an atom into 
the specified unit supercell if its coordinates are outside of the unit 
supercell.  For example, if the coordinate of the silicon atom is
chosen to be $(0,0,0)$, then the atomic positions shown on the left
side of Figure~\ref{sih4atom} are equivalent to those shown
on the right side of the figure.
\begin{figure}[htbp]
\hfill
\begin{center}
\subfigure[Atomic positions outside of the unit supercell]
{
    \includegraphics[width=0.3\textwidth]{sih41.png}
    \label{sih41}
}
\subfigure[Atomic positions inside of the unit supercell]
  {
    \includegraphics[width=0.3\textwidth]{sih42.png}
    \label{sih42}
  }
\end{center}
\caption{Equivalent rendering of the atomic positions of Si and H in a
silane molecule}
\label{sih4atom}
\end{figure}


\subsection{Kinetic Energy Cutoff}  
In order to discretize the Kohn-Sham equations with an appropriate number
of planewave basis functions, one must specify a kinetic energy cutoff
$E_{cut}$ through 
\begin{verbatim}
mol = set(mol,'ecut',cutoff_value);
\end{verbatim}
where {\tt cutoff\_value} is a user supplied number in the unit of Rydberg (Ryd).

This energy cutoff defines a maximum magnitude limit for the wave numbers 
$g_j$ allowed in the planewave expansion of a single-particle wavefunction 
$\psi(r)$.  That is, when the cutoff is set to $E_{cut}$, $\psi(r)$ is 
expressed as
\begin{equation}
\psi(r) = \sum_{j=1}^{n_g} c_j e^{i g_j^T r}, 
 \label{fexpand}
\end{equation}
where $n_g$ is the number of all wave numbers $g_j$ that satisfy
\begin{equation}
||g_j ||^2 < 2 E_{cut}, \label{ecut}
\end{equation}

The frequency cutoff $E_{cut}$ also allows us to determine 
the minimum number of sampling grid points ($n_1, n_2, n_3$) we need
to represent $\psi(r)$ in real space without causing aliasing effects
\cite{kssolv}.

Clearly, the larger the $E_{cut}$ value we use, the larger the number of 
planewave basis functions and real space sampling grid points are required 
in the representations of $\psi(r)$. The use of a large number of planewave 
basis leads to a more accurate finite dimensional approximation to the
continuous Kohn-Sham equations.  Unfortunately, this also leads to a 
high computational cost (in term of both time and storage) required to solve 
the finite dimensional Kohn-Sham equations. 

The optimal choice of $E_{cut}$ is system dependent.  The choice also depends 
on the pseudopotential function used.  However, for the purpose of
understanding the numerical method for solving a finite dimensional KS 
equation, it is sufficient to set $E_{cut}$ to a relatively small number,
e.g., $E_{cut} = 25 Ryd$.  One can experimenting with different choices
of $E_{cut}$ to see how the minimum total energy changes as one increase
the number of plane wave basis in the discretization.
%
{\footnotesize
\begin{figure}[htbp]
\begin{center}
\begin{boxedverbatim}
%
% 1. construct atoms
%
a1 = Atom('Si');
a2 = Atom('H');
alist = [a1; a2; a2; a2; a2];
%
% 2. set up primitive cell (Bravais lattice)
%
C = 10*eye(3);
%
% 3. define the coordinates the atoms
%
coefs = [
 0.0     0.0      0.0
 0.161   0.161    0.161
-0.161  -0.161    0.161
 0.161  -0.161   -0.161
-0.161   0.161   -0.161
];
xyzmat = coefs*C';
%
% 4. Configure the molecule
%
mol = Molecule();
mol = set(mol,'Blattice',C);
mol = set(mol,'atomlist',alist);
mol = set(mol,'xyzlist' ,xyzmat);
mol = set(mol,'ecut', 25);  % kinetic energy cut off
mol = set(mol,'name','SiH4');
\end{boxedverbatim}
\caption{Setting up the $Silane$ molecule}
\label{sih4source}
\end{center}
\end{figure}
}
%
\subsection{Checking the Attributes of an {\tt Molecule} Object}
After all attributes of an {\tt Molecule} object {\tt mol} have been
set, one can view all attributes by simply typing {\tt mol} (without
semicolon) at the MATLAB prompt without a semicolon at the end. 
For example, typing {\tt mol} after 
running {\tt sih4\_setup} gives the output shown in Figure~\ref{sih4display}.
{\footnotesize
\begin{figure}[htbp]
\begin{center}
\begin{boxedverbatim}
>> mol
Molecule: SiH4
   primitive cell:
      1.000e+01     0.000e+00     0.000e+00
      0.000e+00     1.000e+01     0.000e+00
      0.000e+00     0.000e+00     1.000e+01
   sampling size: n1 = 32, n2 = 32, n3 = 32
   atoms and coordinates: 
       1   Si      0.000e+00      0.000e+00      0.000e+00
       2    H      1.610e+00      1.610e+00      1.610e+00
       3    H     -1.610e+00     -1.610e+00      1.610e+00
       4    H      1.610e+00     -1.610e+00     -1.610e+00
       5    H     -1.610e+00      1.610e+00     -1.610e+00
   number of electrons  : 8
   spin type            : 1
   kinetic energy cutoff: 2.500e+01
\end{boxedverbatim}
\caption{Displaying the attributes of the Silane molecule}
\label{sih4display}
\end{center}
\end{figure}
}

Note that once the list of atoms is specified, the {\tt Molecule}
object automatically calculates the number of valence electrons
contained in the molecular system.  In the case of the $SiH_4$
molecule, there are 8 valence electrons. Four of them are contributed
by the Si atom in the 2s and 2p atomic orbitals. The four hydrogen atoms each
contributes an electron.  Currently, KSSOLV does not distinguish 
the spin orientations of valence electrons in molecules or solids. 
This is indicated by the {\tt spin type} attribute in the {\tt Molecule}
object, which is set to 1 in Figure~\ref{sih4display}.  
What that means is that for this particular system, electrons 
are considered in pairs, and the number of occupied states (Kohn-Sham
wavefunctions) is half of the number of valence electrons.

For systems that do not contain atoms such as electron quantum dots
that contain electrons confined by an external potential. The number 
of electrons {\tt nel} must be specified explicitly through a statement 
like
\begin{verbatim}
   mol = set(mol,'nel',4);
\end{verbatim}
In this case, one may also set the {\tt spin type} to 2 by 
\begin{verbatim}
   mol = set(mol,'nspin',2)
\end{verbatim}
so that electrons will not be considered in pairs.
%
\section{Solving the Kohn-Sham Equations} \label{sec:solve}
Once an {\tt Molecule} object is set up properly, computing
the electron wavefunctions associated with the minimum total
energy of the molecular system can be performed in KSSOLV by calling 
one of the functions listed in Table~\ref{methodlist}.  We will
describe these methods in detail in chapter~\ref{chap:methods}.
%
\begin{table}[htbp]
\center
\begin{tabular}{|c|c|} \hline
 function       &   Description \\ \hline
{\tt scf.m}     &   Self consistent field iteration \\
{\tt chebyscf.m}&   Chebyshev filtered self consistent field iteration \\
{\tt dcm.m}     &   direct constrained minimization (DCM) \\
{\tt trdcm.m}   &   direct constrained minimization  \\
                &   with flexible trust region parameters \\
{\tt trdcm1.m}  &   direct constrained minimization  \\
                &   with a fixed trust region parameter \\
\hline
\end{tabular}
\caption{Methods developed for Solving the KS equation in KSSOLV}
\label{methodlist}
\end{table}
%

The simplest way to use the {\tt scf} function which contains an
implementation of the self-consistent field (SCF) iteration 
with charge mixing \cite{pulay:80,pulay:82,kerker} acceleration is:
\begin{verbatim}
>> [Etotvec, X, vtot, rho] = scf(mol);
\end{verbatim}
The function call takes {\tt mol} as its only input and produces the
following outputs:
\begin{enumerate}
\item {\tt Etotvec} is an array of total energies computed at each SCF
  iteration;
\item {\tt X} is a {\tt Wavefun} object that contains the approximate
single-particle wavefunctions returned from the SCF
calculation. The details on the structure of the {\tt Wavefun} class
are explained in Chapter~\ref{chap:class};
\item {\tt vtot} is a 3D array that contains the total potential associated
with the returned single-particle wavefunctions of the {\tt mol} object.
\item {\tt rho} is a 3D array that contains the electron charge associate
with the returned single-particle wavefunctions of the {\tt mol} object.
\end{enumerate}  

All functions listed in ~\ref{methodlist} allow additional options to be
passed in through an option structure that can be defined by calling
the function {\tt setksopt}.  

Calling {\tt setksopt} without any argument simply creates an option
structure with default parameter settings. For example,
\begin{verbatim}
   opts = setksopt;
\end{verbatim}
creates an option structure {\tt opt} that contains the following fields
\begin{verbatim}
   opts = 

          verbose: 'on'
       maxscfiter: 10
       maxdcmiter: 10
       maxinerscf: 3
        maxcgiter: 10
           scftol: 1.0000e-08
           dcmtol: 1.0000e-08
            cgtol: 1.0000e-06
          mixtype: 'anderson'
           mixdim: 9
          betamix: 0.8000
            brank: 1
               X0: []
             rho0: []
\end{verbatim}

Each field can be modified by simply resetting the value of the field 
through an assignment statement. For example, to change the maximum
number of SCF iterations allowed, we can simply use
\begin{verbatim}
   opts.maxscfiter = 20.
\end{verbatim}
This parameter can also be modified by calling {\tt setksopt} again with
appropriate argument pairs. For example,
\begin{verbatim}
   opts = setksopt(opts,'maxscfiter',20);
\end{verbatim}
achieves the same effect.  The advantage of calling {\tt setksopt} is
that the function checks the validity of the structure field and its
value.

More information about the options and default values for all methods listed
in Figure~\ref{methodlist} can be found in Chapter~\ref{chap:methods}. 

\section{Understanding the Output from KSSOLV}
By default, both the SCF and the DCM methods implemented in KSSOLV
print out a bunch of information on the screen.  A typical SCF run
produces diagnostic output like the one shown in Figure~\ref{sih4scf}.
After some initialization, the code uses the either the 
LOBPCG algorithm \cite{lobpcg} to be described later in section 4 (which is
the default) or some other subspace iteration based algorithms to solve 
a sequence of linear eigenvalue problem. In the case of the silane molecule 
($SiH_4$), there are 4 electron pairs. Thus 4 smallest eigenvalues of a 
fixed Hamiltonian are approximated at each SCF iteration.  The diagnostic 
output shows both the eigenvalue approximation and the estimated error 
associated with these eigenvalues and their corresponding eigenvectors. 
At the end of each SCF iteration,
the code also prints out the total energy evaluated at the current
wavefunction approximation, the residual error associated with 
eigenvalue and wavefunction approximations (to the solution of 
the Kohn-Sham equations (\ref{ev1})), and the lack of self-consistency in the
total potential.

\begin{figure}[htbp]
\begin{center}
\begin{boxedverbatim}
[Etotvec, X, vtot, rho] = scf(mol);
initialization...
Beging SCF calculation...
LOBPCG iter =   1
eigval( 1) =   7.195e+00, resnrm =   3.277e+00
eigval( 2) =   7.368e+00, resnrm =   3.347e+00
eigval( 3) =   7.551e+00, resnrm =   3.233e+00
eigval( 4) =   7.703e+00, resnrm =   3.239e+00

LOBPCG iter =   2
eigval( 1) =   5.097e-01, resnrm =   8.297e-01
eigval( 2) =   6.200e-01, resnrm =   8.719e-01
eigval( 3) =   6.912e-01, resnrm =   9.045e-01
eigval( 4) =   9.038e-01, resnrm =   9.130e-01
...
SCF iter  1:
norm(vout-vin) =  5.807e+00
Total energy   = -5.8719608972592e+00
 resnrm =   1.857e-02
 resnrm =   2.161e-02
 resnrm =   2.161e-02
 resnrm =   2.162e-02
...
\end{boxedverbatim}
\caption{SCF output}
\label{sih4scf}
\end{center}
\end{figure}


\section{Visualizing the Results}
The electron charge density, total potential and single particle wavefunctions
can be easily visualized using either the MATLAB's isosurface rendering
function {\tt isosurface} or the volume rendering function written by Joe Conti, which 
we also include in KSSOLV.  To see the isosurface of the charge density {\tt
 rho}, one can simply type
\begin{verbatim}
>> isosurface(rho).
\end{verbatim}

The {\tt scf}, {\tt dcm} and {\tt trdcm} functions in KSSOLV return
a charge density defined within the specified unit supercell even 
though the atomic coordinates defined during the initial setup may 
not lie within that cell.  To see the a charge density in the original 
atomic coordinates, one may use the MATLAB {\tt fftshift} function first
to shift density before rendering. That is, one may use
\begin{verbatim}
>> isosurface(fftshift(rho))
\end{verbatim}
to view the isosurface of the charge density.

Figure~\ref{sih4rho} shows the isosurface rendering of the charge density of
the $SiH_4$ molecule associated with the atomic positions defined in 
Figure~\ref{sih4source} and displayed in Figure~\ref{sih41}.
%
\begin{figure}[h!]
\begin{center}
\includegraphics[width=0.5\textwidth]{sih4_rho.png}
\end{center}
\caption{The computed charge density of the $SiH_4$ molecule.}
\label{sih4rho}
\end{figure}
%
Similarly, one can use
\begin{verbatim}
>> view(3);
>> vol3d('cdata',fftshift(rho));
\end{verbatim}
to view a volume rendering of {\tt rho}.
\begin{figure}[h!]
\begin{center}
\includegraphics[width=0.5\textwidth]{volsih4.png}
\end{center}
\caption{A volume rendering of the computed charge density of the $SiH_4$ molecule.}
\label{sih4rhovr}
\end{figure}

\chapter{KSSOLV Classes} \label{chap:class}
In this chapter, we will describe all classes defined in KSSOLV in 
detail.  In particular, we will examine the structure of these classes and
methods that can be invoked. We will also discuss overloaded operators that 
can be used directly on objects that belong to certain classes.  

In MATLAB, a user-defined class is identified by a directory whose name 
begins with the '@' symbol.  Therefore, once KSSOLV is installed,
one should find the following directories in KSSOLV's root directory {\tt kssolv/}:
\begin{verbatim}
@Atom/
@Molecule/
@Ham/
@Wavefun/
@FreqMask/
\end{verbatim}

Each @-prefixed directory contains an M-file with the same name (excluding 
the @ symbol).  This M-file is a {\em constructor} that can be used to 
instantiate a specific KSSOLV object that belongs to a certain class.
The function defined in this M-file can be called without any input
argument to create a null object that needs to be refined later by calling 
the {\tt set} method.  For example, the following call creates a null {\tt Atom}object.
\begin{verbatim}
a = Atom();
\end{verbatim}
One may also call the constructor with one or a few arguments to initialize 
the object when it is created.  For example, the following call creates
an Hamiltonian associated with a predefined {\tt Molecule} object {\tt mol}
using a predefined charge density {\tt rho}.
\begin{verbatim}
H = Ham(mol,rho);
\end{verbatim}

Each class contains a number of attributes specific to that class. 
These attributes must be defined properly before an object belonging to that
class is used.  When these attributes are not initialized upon the initial
construction, they can be set by using the {\tt set} method defined
for that class.  For example, if an {\tt Atom} object {\tt a} is created
by calling the constructor with no input argument (as illustrated above),
we can use 
\begin{verbatim}
a = set(a,'symbol','H'); 
\end{verbatim}
to specify that {\tt a} is an hydrogen atom. Here the string 'symbol' 
is a keyword that represents the name of a specific attribute, which in
this case refers to the chemical symbol of an atom. The user
provided string 'H' is a value assigned to that attribute.  

Note that not all attributes of an object need to be set before the
object can be used in subsequent calculations. Setting some key attributes 
will ensure that other attributes of the same object are properly set 
automatically.  In fact, for an {\tt Atom} object, the only attribute
that needs to be set is either the atomic symbol 'symbol', or the atomic 
number 'anum'. All other attributes of the atom, which we will examine below, 
will be defined automatically once 'anum' or 'symbol' is set.

All attributes of an object can be retrieved by using the {\tt get} method
defined for the class to which the object belongs. For example,
\begin{verbatim}
nvel = get(a,'vnum');
\end{verbatim}
retrieves the number of valence electrons in atom {\tt a}, and assigns that
number to the variable {\tt nvel}.

The essential attributes of an object can also be displayed by typing
the name of object at the MATLAB prompt without a semicolon. 
For example, typing 
\begin{verbatim}
>> a = Atom('Si')
\end{verbatim}
shows
\begin{verbatim}
atomic symbol: Si
atomic number: 14
number of valence electrons: 4
\end{verbatim}

\section{Atom}
Atoms are the basic constituents of molecules and solids. Therefore, it
is natural to define an {\tt Atom} class in KSSOLV so that we can easily 
create atom objects and use them to construct molecules and bulk systems.
An atom can be declared in KSSOLV as an {\tt Atom} object using an {\tt Atom} 
constructor. For example
\begin{verbatim}
a = Atom('Si'); 
\end{verbatim}
or
\begin{verbatim}
a = Atom(14);
\end{verbatim}
defines a silicon atom object named {\tt a}.  Note that an atom can be 
defined by either its chemical symbol or by its atomic number. An 
atom object contains a number of attributes listed and described
in Table~\ref{atomattr}. Most of these attributes are properly set
once the chemical symbol or the atomic number of the atom is initialized.
They are mainly used by KSSOLV internally to construct ionic pseudopotentials.

\begin{table}[htbp]
\center
\begin{tabular}{|c|l|c|} \hline
attribute name &  description          & data type  \\ \hline
'symbol' & chemical symbol             & string  \\
'anum'   & atomic number               & integer \\
'venum'  & number of valence electrons & integer \\
%'iloc'   &  xxx                        & integer \\
%'occs'   &  xxx                        & integer \\ 
%'occd'   &  xxx                        & integer \\
%'iso'    &  xxx                        & integer \\ 
%'ic'     &  xxx                        & integer \\
%'isref'  &  xxx                        & integer \\
%'ipref'  &  xxx                        & integer \\
%'idref'  &  xxx                        & integer \\ 
\hline
\end{tabular}
\caption{Attributes of an {\tt Atom} object }
\label{atomattr}
\end{table}

All these attributes are accessible through the {\tt get} method defined for
the {\tt Atom} class. However, a user would rarely need to retrieve attributes
other than the atomic number and the number of valence electrons.

\section{Molecule}
The keyword {\tt Molecule} is used in KSSOLV broadly to refer to both
a single molecule and a periodic (bulk) system.  The {\tt Molecule}
class is created to encapsulate both the physical attributes of a molecule 
and the planewave discretization parameters that must be 
defined before the Kohn-Sham equations associated with a particular 
{\tt Molecule} object can be solved. 

To create an {\tt Molecule} object, one typically uses
\begin{verbatim}
mol = Molecule();
\end{verbatim}
to first construct an empty object called {\tt mol} (a user defined variable 
name). This call simply sets up the required data structure that would be 
used to describe various attributes of {\tt mol}.
The actual attributes of {\tt mol} such as the number and type of atoms in 
this molecule, the size and shape of the supercell that contains the molecule 
etc. can be defined through the {\tt set} method.
\begin{verbatim}
mol = set(mol,attrname,attrvalue);
\end{verbatim}
where the input argument {\tt attrname} is a predefined string associated
with the {\tt Molecule} class that specifies the name of a particular 
attribute, and {\tt attrvalue} is a user supplied quantity that will 
be stored in {\tt mol}.  Table~\ref{molattr} lists the all attributes of an 
{\tt Molecule} object. Some of these attributes such as the name of the
molecule and the external potential are optional. Some attributes such as 
the number of sampling grid points {\tt n1}, {\tt n2}, {\tt n3} will be 
properly initialized after other attributes such as {\tt ecut} are defined.
%
\begin{table}[htbp]
\center
\begin{tabular}{|c|l|l|} \hline
attribute name &  purpose & value type  \\ \hline
'name'      & The name of the Molecule & string \\
'supercell'     & Bravais lattice box      & $3 \times 3$ matrix\\
'atomlist'  & list of atoms & array of {\tt Atom} objects \\
'xyzlist'   & list of atomic coordinates & an $n_a \times 3$ matrix \\
'vol'       & The volume of the primary (super)cell & \\ 
            & that contains the molecule            & float \\ 
'natoms'    & number of atoms in the primary cell & integer \\
'ecut'      & planewave energy cutoff & floating point scalar \\
'n1'        & the number of sampling points along the $x$-axis & integer \\
'n2'        & the number of sampling points along the $y$-axis & integer \\
'n3'        & the number of sampling points along the $y$-axis & integer \\
'nspin'     & the spin type  & 1 or 2 \\
'nel'       & the number of valence electrons & integer \\
'vext'      & external potential & 3D array of size $n1\times n2 \times n3$\\
\hline
\end{tabular}
\caption{Attributes to be set in an {\tt Molecule} object }
\label{molattr}
\end{table}

The datatypes associated with most of the attributes are straightforward. A
few require further explanation.   The {\tt 'supercell'} attribute
defines the shape and size of the unit supercell that contains 
the basic atomic constituents of a single molecule or solid.
In KSSOLV, the super cell is described by a $3 \times 3$ matrix. Each row of 
this matrix defines a lattice vector emanating from the origin. For example, 
\begin{verbatim}
mol = set(mol,'supercell',10*eye(3));
\end{verbatim}
sets the unit supercell of {\tt mol} to a $10\times 10 \times 10$ cube 
whose three basic lattice vectors are parallel to the $x$, $y$ and $z$ 
axes respectively.  Atomic units (Bohr) are used in the definition of the 
unit cell.

The 'atomlist' attribute is defined to be an array of {\tt Atom} objects.
An example of how to construct such an array has been shown in 
Chapter~\ref{chap:start}.
The coordinates associated with $n_a$ atoms listed in 'atomlist' must be
defined in an $n_a \times 3$ array that contains the x, y, z coordinates 
(in atomic units) for each atom. 

Currently, KSSOLV makes no distinction between the wavefunctions associated 
with electrons occupying the same orbitals. In this case, the spin type 
is normally set to {\tt nspin=1}.

However,  KSSOLV can also be used to solve the Kohn-Sham equations for 
electron-only systems. In this case, electron wavefunctions are not 
associated with any atoms, and they must be treated differently, which 
indicated in KSSOLV by setting {\tt nspin} to 2.

\section{Ham}
A properly defined {\tt Molecule} object {\tt mol} can be used to 
initialize a KS Hamiltonian.  The initialization can be done by simply 
calling the constructor
\begin{verbatim}
H = Ham(mol);
\end{verbatim}
The constructed {\tt Ham} object should contain all information
one would need to solve the Kohn-Sham equations. 
Although the KS Hamiltonian $H(X)$ is treated as a matrix in 
equation (\ref{kseq}), it is not stored as a matrix in KSSOLV. 
Instead, the {\tt Ham} class keeps $L$, $\hat{V}_{ion}$ and total 
potential $V_{tot} = \hat{V}_{ion} +\mbox{Diag}(L^{\dagger}\rho)+
\mbox{Diag}(\mu_{xc}(\rho))$ 
as separate attributes.  This separation makes it easy to update 
the Hamiltonian in both the SCF and DCM calculations.  Table~\ref{hamattr}
shows all other important attributes of an {\tt Ham} object.
%
\begin{table}[htbp]
\center
\begin{tabular}{|c|l|l|} \hline
attribute name &  purpose & value type  \\ \hline
'gkk'   & \begin{minipage}[t]{4in}
          The magnitude square of wave numbers associated with the
          kinetic energy that satisfy the cutoff constraint (\ref{ecut}) 
          \end{minipage}
        & 1D array \\
'idxnz' & \begin{minipage}[t]{4in}
          positions of the reciprocal lattice points associated with 
          all entries of {\tt gkk} within a 3D array 
          \end{minipage}
        & 1D array \\
'vtot'  & total local potential & 3D array \\
'vion' & local ionic pseudopotential in real space & 3D array \\
'wqmat'& Fourier representation of the non-local pseudopotential 
       & \begin{minipage}[t]{.6in} 2D array (matrix)\end{minipage} \\
'vext' & external potential & 3D array \\ 
'vnp'  & charge dependent portion of the potential & 3D array \\
\hline
\end{tabular}
\caption{Attributes to be set in an {\tt Molecule} object }
\label{hamattr}
\end{table}

The kinetic component of a {\tt Ham} object is represented and stored 
as a 1D array ({\tt gkk}) of integers that represent the magnitude squares 
of the wave numbers that satisfy the energy cutoff constraint (\ref{ecut}). 
These values correspond to the eigenvalues of a Laplacian 
operator (with periodic boundary conditions) sampled in the reciprocal
(frequency) space. Only wave numbers (or frequencies)
that satisfy the kinetic energy cutoff set in {\tt mol} are stored.
There numbers correspond to reciprocal space lattice points that lie 
within a sphere with a radius of $\sqrt{2E_{cut}}$.
The locations of these lattice points in a 3D array of all lattice
points (including those outside of the cutoff sphere) are kept 
in a separate array {\tt idxnz} so that wave numbers associated with
reciprocal space lattice points outside of the kinetic energy cutoff sphere
are not used or stored.

The local ionic potential $\hat{V}_{ion}$ is also constructed and stored 
as a 3D array at initialization.  The construction process involves
loading the atomic pseudopotential stored in the directory {\tt Pseudopot} 
and summing them up using the scheme suggested by Kleinman and Bylander 
\cite{kbpseudo}.  KSSOLV provides Troullier-Martins \cite{tmpseudo} atomic 
pseudopotentials associated with all atoms listed in the first four
rows of the periodic table. Some elements in higher row of the
table are also provided.

The determination of both the Hartree and exchange correlation
potential requires the availability of the charge density $\rho$
which is in turn a function of the wavefunctions to be computed.
Since a good approximation to the desired wavefunctions is not available
at initiation, the initial $\rho$ is computed in KSSOLV by combining
atomic charge densities associated with each atom in the {\tt mol}
object \cite{kbpseudo}.

Once a {\tt Ham} object has been defined, one can retrieve various
attributes of the object through the {\tt get} function, e.g.,
\begin{verbatim}
vt = get(H,'vtot');
\end{verbatim}
returns the total potential from a {\tt Ham} object named {\tt H}
and assigns it to a user defined variable {\tt vt}. This potential
can be used and updated in a subsequent SCF or DCM calculation. An updated
{\tt vt} can be passed into {\tt H} by using the {\tt set} function
\begin{verbatim}
H = set(H,'vtot',vt);
\end{verbatim}

\section{Wavefun}
We created a special class {\tt Wavefun} in KSSOLV to specify one
or a set of wavefunctions rather than representing them by a MATLAB vector
or matrix.  The rationale for creating such a class is mainly
to reduce the amount of codes a user would have to write to 
reshape a 3D array (which is the data structure required for a 3D FFT call)
into a 1D vector (which is the most convenient data structure
for linear algebra operations) and vice versa.  Representing 
a wavefunction as a {\tt Wavefun} object also makes it easy to 
perform the multiplication of a KS Hamiltonian with a wavefunction
and basic linear algebra operations between  wavefunctions
through operator overloading.  

In KSSOLV, a {\tt Wavefun} object stores the Fourier coefficients
of one or several wavefunctions that are expanded in a planewave 
basis defined by (\ref{fexpand}).  The object can be constructed 
using either a noncompact scheme or a compact scheme.  In a noncompact 
representation, a {\tt Wavefun} object {\tt X} can be constructed 
through the command
\begin{verbatim}
X = Wavefun(psi),
\end{verbatim}
where {\tt psi} is a MATLAB 3D array if {\tt X} represents a single
wavefunction, or a cell array that contains a list of 3D arrays if
{\tt X} represents a set of wavefunctions.

Under the compact scheme, a {\tt Wavefun} object stores 
only the Fourier coefficients ($c_{i,j}$) of a truncated expansion of
a single wavefunction $\psi(r)$ or a set of wavefunctions 
$\{\psi_i(r)\}_{i=1}^{n_e}$.  These coefficients are associated with wave 
numbers ($g_j$) that satisfy the cutoff constraint (\ref{ecut}), and stored 
contiguously as a MATLAB cell array of size $n_g$ by {\tt nocc}, 
where $n_g$ is the number of these Fourier expansion coefficients, 
and {\tt nocc} is the number of occupied states (electron pairs).  
All other Fourier coefficients that may appear in a full planewave 
expansion are considered zeros, thus never used or stored.  
Storing the nonzero Fourier expansion coefficients contiguously in 
a 1D array makes linear algebra operations on {\tt X} more efficient. 
However, to perform 3D FFT operations on {\tt X}, we must place these 
coefficients in a 3D array. Hence, in the compact scheme, we must also 
record the locations of these Fourier coefficients in a 3D array, 
which are the locations of the reciprocal lattice points associated with 
the corresponding wave numbers in the planewave basis.  These locations 
are stored in a separate array which is labeled as the '{\tt idxnz}' 
attribute of the {\tt Wavefun} object.  Recall that the same strategy 
is used to store the magnitude squares of the wave numbers and their 
locations in a 3D array for the {\tt Ham} object.  In signal processing, 
functions that are expanded by planewaves up to certain frequency or wave 
number are called {\em bandlimited} functions.  

The size of the 3D array that is used to hold the Fourier coefficients
when an FFT is performed is specified by the '{\tt n1}', '{\tt n2}' and 
'{\tt n3}' attributes of the {\tt Wavefun} object.  These numbers also 
give the dimension of the corresponding wavefunction in real space.  
They should be the same as those specified in a {\tt mol} object.

The constructor for a {\tt Wavefun} object can be called in a number of
ways.  In addition to creating a null object by using,
\begin{verbatim}
X = Wavefun();
\end{verbatim}
we may also create a null object of certain size by using
\begin{verbatim}
X = Wavefun(n1,n2,n3);
\end{verbatim}
where {\tt n1}, {\tt n2} and {\tt n3} are predefined integers.
The following use of the constructor creates a {\tt Wavefun} object
using a noncompact scheme.
\begin{verbatim}
X = Wavefun(Q);
\end{verbatim}
where {\tt Q} is either a 3D array or a cell array that contains a
number of 3D arrays. These 3D arrays should contain the Fourier 
expansion coefficients of one or several wavefunctions.

Finally, a fixed size Wavefun object in compact scheme can be constructed by
using
\begin{verbatim}
X = Wavefun(Q,n1,n2,n3,idxnz);
\end{verbatim}
where $Q$ is a 1D array that stores the Fourier expansion coefficients
continuously and {\tt idxnz} is a 1D array that describes the locations
of these coefficients in a 3D array of size $n1 \times n2 \times n3$.

In the following section, we will see that all major matrix operations 
have been overloaded for {\tt Wavefun} objects for both the compact and
noncompact schemes.  KSSOLV also provides a utility function {\tt genX0}
that allows one to easily construct initial {\tt Wavefun} objects for the 
SCF or DCM calculations. To generate a set of random bandlimited 
wavefunctions using the kinetic energy cutoff specified in a 
{\tt Molecule} object {\tt mol}, one can simply use the command
\begin{verbatim}
X = genX0(mol).
\end{verbatim}

Converting a {\tt Wavefun} object {\tt X} to a 3D array (or a list of 3D
arrays) is straightforward when {\tt X} is constructed using a noncompact 
scheme. The following command
\begin{verbatim}
X3D = get(X,'psi')
\end{verbatim}
returns the wavefunctions as a cell array {\tt X3D} of 3D arrays. 
Although rarely needed when using KSSOLV, the following lines of codes show
how the same conversion can be accomplished for an {\tt X} constructed
using a compact representation scheme
\begin{verbatim}
n1  = get(X,'n1');
n2  = get(X,'n2');
n3  = get(X,'n3');
psi = get(X,'psi');
idx = get(X,'idxnz');
X3D = zeros(n1,n2,n3);
X3D(idx) = psi{1}.
\end{verbatim}

\section{FreqMask}
The {\tt FreqMask} class is created to define an object
that stores the wave numbers (frequencies) that appear
in (\ref{fexpand}) and their magnitudes. The three components of 
each wave number $g_j$ are stored in three separate arrays {\tt gkx}, 
{\tt gky} and {\tt gkz}.  These arrays are attributes of a 
{\tt FreqMask}.  They are used in the pseudopotential calculation
({\tt getvion.m}, {\tt getwq.m}) and a number of other KSSOLV functions.
Only the wave numbers that satisfy the energy cutoff criterion 
(\ref{ecut}) are defined and stored when a {\tt FreqMask} object
is constructed.  The number of wave numbers that satisfy (\ref{ecut}) 
is stored as the attribute {\tt ng}.

A {\tt FreqMask} object can be constructed by calling
\begin{verbatim}
   gm = FreqMask(mol).
\end{verbatim}
In this case, the energy cutoff defined in the {\tt Molecule} object
{\tt mol} is used to determine which wave numbers should be set 
in {\tt gkx}, {\tt gky} and {\tt gkz}. The constructor
of the object also calculates and stores the magnitude square of 
each Fourier expansion coefficient that satisfies (\ref{ecut}) 
in an separate array {\tt gkk}.  Because such a quantity is used 
frequently in the numerical procedure for solving the Kohn-Sham 
equations we store it in a {\tt FreqMask} object instead of computing 
it repeatedly from {\tt gkx}, {\tt gky} and {\tt gkz} each time
it is used.

Alternatively, one can pass in a different energy cutoff and 
create a {\tt FreqMask} object by using
\begin{verbatim}
   gm = FreqMask(mol,ec),
\end{verbatim}
where {\tt ec} is a user provide energy cutoff value.

\section{Operator Overloading}
Because the Kohn-Sham Hamiltonian $H(X)$ and wavefunction $X$ are viewed
as matrices in the algorithms to be described in the next chapter, 
it is desirable to allow {\tt Hamilt} and {\tt Wavefun} objects to be 
manipulated in KSSOLV as if they are matrices.  This feature is made possible
in KSSOLV by overloading some basic algebraic operations for
a {\tt Wavefun} object. These overloaded operations are listed in 
Table~\ref{wvops}.  One should be careful about the use of 
some of these operators. For example, since the wavefunctions used in the SCF
and DCM calculation all have the same dimension, the multiplication
operator {\tt *} is never used between two {\tt Wavefun}
objects except when the first {\tt Wavefun} object is transposed
or conjugate transposed, i.e., it is valid to perform {\tt x'*y}
or {\tt x.'*y}, and the multiplication returns a standard MATLAB
matrix object.  The overloaded multiplication operator {\tt *} for 
{\tt Wavefun} objects allows the second operand to be a standard matrix 
object with proper dimension. The result of the multiplication is a 
{\tt Wavefun} object.
%
\begin{table}[htbp]
\center
\begin{tabular}{|l|l|} \hline
Operations     &  Description \\ \hline
{\tt x + y}    & Add two wavefunctions \\
{\tt x - y}    & Subtract one wavefunction from another \\
{\tt x * y}    & Multiply two wavefunctions and return a matrix\\
{\tt x * a}    & Multiply several wavefunctions with a matrix\\
{\tt x .* y}   & Element-wise multiplication of two wavefunctions\\
{\tt x .\ y}   & Element-wise division of two wavefunctions\\
{\tt x'}       & Complex conjugate transpose of a wavefunction\\
{\tt x.'}      & Transpose of a wavefunction\\
{\tt [x y]}    & Horizontal concatenation of several wavefunctions\\
{\tt x(:,i:j)} & Subscripted reference of wavefunctions\\
\hline
\end{tabular}
\caption{Overload operations for {\tt Wavefun} objects in KSSOLV}
\label{wvops}
\end{table}

The multiplication operator is also overloaded for the {\tt Hamilt} class
so that the multiplication of a {\tt Hamilt} object {\tt H} and 
a {\tt Wavefun} object {\tt X} can be accomplished in KSSOLV
by a simple expression
\begin{verbatim}
Y = H*X,
\end{verbatim}
which hides all the complexity of the operation from
the user.


\chapter{KSSOLV Methods} \label{chap:methods}
In this chapter, we will briefly describe the numerical methods employed in
KSSOLV to obtain an approximate solution to the Kohn-Sham equations 
(\ref{ev1}).  We begin by discussing the planewave
discretization scheme that turns the continuous nonlinear problem 
into a finite dimensional problem.  The finite dimensional problem 
is expressed as a matrix nonlinear eigenvalue problem.  We present 
two different approaches to solving the matrix nonlinear eigenvalue problem 
in sections~\ref{sec:scf} and \ref{sec:dcm}. 

\section{Planewave Discretization}\label{sec:pwdiscretize}
To solve the minimization problem (\ref{ksmin}) or the Kohn-Sham equation 
(\ref{ev1}) numerically, we must first discretize the continuous problem.
Standard discretization schemes such as finite difference,
finite elements and other basis expansion (Ritz-Galerkin) methods 
\cite{ritz} all have been used in practice.  The discretization scheme we have 
implemented in the current version of KSSOLV is a Ritz type of method that 
expresses a single electron wavefunction $\psi(r)$ as a linear combination 
of planewaves $\{e^{-ig_j^Tr}\}$, where $g_j \in \mathbb{R}^3$ ($j =
1,2,...,K$) are wave numbers (frequency vectors) arranged in a 
lexicographical order.  
The planewave basis is a natural choice for studying periodic systems such 
as solids.  It can also be applied to non-periodic structures (e.g., molecules)
by embedding these structures in a ficticious supercell \cite{payne:92} that 
is periodically extended throughout an open domain.  The use of the planewave 
basis has the additional advantage of making various energy calculations in 
density functional theory easy to implement.  It is the most convenient 
choice for developing and testing numerical algorithms for solving 
the Kohn-Sham equations within the MATLAB environment, partly due to the 
availability of efficient fast Fourier transform (FFT) functions. 

It is natural to assume that the potential for $R$-periodic atomistic
systems is a periodic function with a period $R \equiv (R_1, R_2, R_3)$.  
Consequently, we can
restrict ourselves to one canonical period often referred to as the primitive
cell and impose periodic boundary condition on the restricted problem. It
follows from the {\em Bloch}'s theorem \cite{am:76,bloch} that eigenfunctions of the 
restricted problem $\psi(r)$ can be periodically extended to the entire domain 
(to form the eigenfunction of the original Hamiltonian) by using the following formula:
\begin{equation}
\psi(r+R) = e^{i k^T R} \psi(r),
\label{rperiod}
\end{equation}
where $k=(k_1,k_2,k_3)$ is a frequency or wave vector that belongs to a primitive cell 
in the reciprocal space (e.g., the first {\em Brillouin} zone \cite{am:76}). 
If the $R$-periodic system spans the entire infinite open domain, the set of 
$k$'s allowed in (\ref{rperiod}) forms a continuum in the first Brillouin zone. 
That is, each 
$\psi(r)$ generates an infinite number of eigenfunctions for the periodic 
structure.  It can be shown that the corresponding eigenvalues form a
continuous cluster in the spectrum of the original Hamiltonian \cite{am:76}. 
Such a cluster is often referred to as an energy band in physics.  
Consequently, the complete set of eigenvectors of $H$ can be indexed by the band
number $i$ and the Brillouin frequency vector $k$ (often referred to as a
$k$-point), i.e., $\psi_{i,k}$. 
In this case, the evaluation of the charge density must first be
performed at each $k$-point by replacing $\psi_i(r)$ in
(\ref{chargeden}) with $\psi_{i,k}$ to yield
\[
\rho_k = \sum_{i=1}^{n_e}| \psi_{i,k}|^2.
\]
The total charge density $\rho(r)$ can then be obtained by integrating over $k$, i.e.,
\begin{equation}
\rho(r) = \frac{|\Omega|}{(2\pi)^3}\int_{BZ} \rho_k(r) dk,
\label{rhointk}
\end{equation}
where $|\Omega|$ denotes the volume of the primitive cell in the first
Brillouin zone.   Furthermore, an integration with respect to $k$ must also 
be performed for the kinetic energy term in (\ref{ksetot}). 

When the primitive cell (or supercell) in real space is sufficiently large,
the first Brillouin zone becomes so small that the integration 
with respect to $k$ can be approximated by a single $k$-point calculation
in (\ref{rhointk}).

To simplify our exposition, we will, from this point on, assume that a 
large primitive cell is chosen in the real space so that no integration with
respect to $k$ is necessary. Hence we will drop the index $k$ in 
the following discussion and use $\psi(r)$ to represent an $R$-periodic
single particle wavefunction. The periodic nature of $\psi(r)$ implies that
it can be represented (under some mild assumptions) by a Fourier series, i.e.,
\begin{equation}
\psi(r) = \sum_{j=-\infty}^{\infty} c_{j} e^{i g_j^T r},
\label{fseries}
\end{equation}
where $c_{j}$ is a Fourier coefficient that can be computed from
\[
c_{j} = \int_{-R/2}^{R/2} \psi(r) e^{-i g_j^T r} dr.
\]

In practice, the exact number of terms used in (\ref{fseries}) is determined
by a kinetic energy cutoff $E_{cut}$. Such a cutoff yields an approximation
\begin{equation}
\psi(r) = \sum_{j=1}^{n_g} c_{j} e^{i g_j^T r}, 
\label{fsumJ}
\end{equation}
where $n_g$ is chosen such that 
\begin{equation}
||g_j ||^2 < 2 E_{cut},
\label{ecutoff}
\end{equation}
for all $j = 1,2, \ldots , n_g$.   
%JCM: Need to check this statement
Although, the value of $n_g$ will depend on many parameters such as the size and type of the system being studied, it is typically an order of magnitude smaller than $n=n_1 \times n_2 \times n_3$. 

Once $E_{cut}$ is chosen, the minimal number of samples of $r$ along each 
Cartesian coordinate direction ($n_1$, $n_2$, $n_3$) required
to represent $\psi(r)$ (without the aliasing effect) can be determined from 
the sampling theorem \cite{nyquist}. That is, we must choose $n_k$ ($k=1,2,3$) 
sufficiently large so that
\begin{equation}
\frac{1}{2}\biggl(\frac{2\pi n_k}{R_k} \biggr) > 2 \sqrt{2E_{cut}}, \label{ecut2}
\end{equation}
is satisfied, i.e., $n_k$ must satisfy 
$n_k > 2R_k \sqrt{2E_{cut}} / \pi$.

We will denote the uniformly sampled $\psi(r)$ by a vector 
$x \in \mathbb{R}^n$, where $n = n_1n_2n_3$ and the Fourier coefficients $c_{j}$ 
in (\ref{fsumJ}) by a vector $c \in \mathbb{C}^{n}$ with zero paddings 
used to ensure the length of $c$ matches that of $x$. If the elements
of $x$ and $c$ are ordered properly, these two vectors satisfy
\begin{equation}
c = F x.  \label{x2c}
\end{equation}
where $F \in \mathbb{C}^{n \times n}$ is a discrete Fourier transform 
matrix \cite{vanloanbook}.

After a sampling grid has been properly defined, the approximation
to the total energy can be evaluated by replacing the integrals
in (\ref{ksetot}) with simple summations over the 
sampling grid.

\section{Pseudopotentials}\label{sec:pseudopot}
To solve the Kohn-Sham equations numerically, the Fourier series expansion
(\ref{fseries}) must be truncated to allow a finite number of terms only. If all
electrons are treated equally, the number of terms required in (\ref{fseries}) 
will be extremely large.  This is due to the observation that the strong 
interaction between a nucleus and the inner electrons of an atom, which
can be attributed to the presence of singularity in $V_{ion}(r)$ at the 
the nuclei position $\hat{r}_j$, must be accounted for by high frequency 
planewaves. However, because the inner
electrons are held tightly to the nuclei, they are not active in terms of
chemical reactions, and they usually do not contribute to chemical bonding or 
other types of interaction among different atoms.  On the other hand, the 
valence electrons (electrons in atomic orbits that are not 
completely filled) can be represented by a relatively small number of 
low frequency planewaves. These electrons are the most interesting ones to 
study because they are responsible for a majority of the physical properties 
of the atomistic system. Hence, it is natural to focus only on these valence 
electrons and treat the inner electrons as part of an ionic core. An 
approximation scheme that formalizes this approach is called the 
{\em pseudo-potential} approximation \cite{phillips58,pk59,yincohen82}.
The details of pseudopotential construction and their theoretical properties
are beyond the scope of this user's guide.  The user should just keep in 
mind that the use of pseudo-potentials allows us to
\begin{enumerate}
\item remove the singularity in $V_{ion}$;
\item reduce the number of electrons $n_e$ in (\ref{ksetot}) and 
      (\ref{chargeden}) to the number of valence electrons;
\item represent the wavefunction associated with a valence electron
      by a small number of low frequency planewaves.
\end{enumerate}

In KSSOLV, the ionic pseudopotential experienced by each single particle
wavefunction is constructed from atomic pseudopotentials that have been
previously constructed and stored in the {\tt Pseudopot} directory.  After
a {\tt Molecule} object {\tt mol} has been properly created, the user
can simply call 
\begin{verbatim}
pseudovar = pseudoinit(mol)
\end{verbatim}
which returns a MATLAB structure {\tt pseudovar} (user defined
variable name).  The {\tt pseudovar} contains various pieces of information
required by KSSOLV to construct the ionic pseudopotential.  In most
cases, a user does not need know what is stored in {\tt pseudovar}. 

The ionic pseudopotential consists of two components.  One of them
is a local component $V_{local}(r)$ that operates pointwise on a single
particle wavefunction $\psi(r)$. The other component is nonlocal. It
is a projection operator that can be represented as
\[
V_{nonlocal}(r) = \frac{1}{\Delta}\sum_{i=1}^{\ell} w_i(r) w_i(r)^{\ast},
\]
where $\Delta$ is a normalization factor, and the rank of the 
projection operator $\ell$ is the total number of quantum number
triplets $(n,l,m)$ associated with all atoms in the system, where
$n$ is the major electron shell number of and $(l,m)$ represent a
angular momentum. 

In KSSOLV, the local ionic potential can be obtained by calling
function {\tt getvion}, i.e.
\begin{verbatim}
vlocal = getvion(pseudovar),
\end{verbatim}
where {\tt vlocal} is a 3D array that contains the potential define
at each real space sampling grid points.  

The nonlocal potential can be computed by calling the function
{\tt getwq}, i.e.
\begin{verbatim}
wqmat = getwq(pseudovar),
\end{verbatim}
where {\tt wq} is a $n_g$ by $\ell$ matrix and $n_g$ is the number of
planewave basis used to represent each single particle wavefunction. 

Both {\tt vlocal} and {\tt wqmat} are computed when a {\tt Ham} object
is constructed initially using
\begin{verbatim}
H = Ham(mol);
\end{verbatim}
for some properly defined {\tt Molecule} object {\tt mol}.  These arrays
are saved as different attributes of {\tt H}.

\section{The finite dimensional Kohn-Sham problem}
If we let $X \equiv (x_1, x_2, ..., x_{n_e}) \in \mathbb{C}^{n\times n_e}$
be a matrix that contains $n_e$ discretized wavefunctions, the approximation 
to the kinetic energy (\ref{ksetot}) can also be expressed by
\begin{equation}
\hat{E}_{kin} = \frac{1}{2}\mbox{trace}(X^{\ast} LX),
\label{ltrace}
\end{equation}
where $L$ is a finite-dimensional representation of the Laplacian operator
in the planewave basis.  Due to the periodic boundary condition imposed 
in our problem, $L$ is a block circulant matrix with circulant blocks that 
can be decomposed as
\begin{equation}
L = F^{\ast} D_g F,  \label{fdf}
\end{equation}
where $F$ is the discrete Fourier transform matrix used in (\ref{x2c}),
and $D_g$ is a diagonal matrix with $|| g_j || ^2$ on the diagonal 
\cite{pjdavis}. It follows from (\ref{x2c}) and (\ref{fdf}) that the 
kinetic energy term (\ref{ltrace} can be evaluated alternatively as
\[
\frac{1}{2}\sum_{\ell=1}^{n_e} \sum_{j=1}^{n_g} ||g_{j} c_j^{(\ell)}||^2.
\]

In the planewave basis, the convolution that appears in the third term 
of (\ref{ksetot}) may be viewed as the $L^{-1}\rho(X)$, where
$\rho(X) = \mbox{diag}(XX^{\ast})$. (To simplify notation, we will drop 
$X$ in $\rho(X)$ in the following.)  However, since 
$L$ is singular (due to the periodic boundary condition), its inverse 
does not exist.  Similar singularity issues appear in the planewave 
representation of the pseudopotential and the calculation
of the ion-ion interaction energy.  However, it can be shown that the 
net effects of these singularities cancel out for a system that 
is electrically neutral \cite{Ihm,pickett}.  Thus, one can
simply remove these singularities by replacing 
$L^{-1}\rho$ with $L^{\dagger}\rho$, where $L^{\dagger}$
is the pseudo-inverse of $L$ defined as
\[
L^{\dagger} = F^{\ast} D^{\dagger}_g F,
\]
where $D^{\dagger}_g$ is a diagonal matrix whose diagonal entries ($d_j$) are
\[
d_j = \left\{
\begin{array}{cc}
||g_j||^{-2} & \mbox{if} \ \ g_j \neq 0;  \\   
0         & \mbox{otherwise}.
\end{array}
\right.
\]
Consequently, the third term in (\ref{ksetot}), which corresponds
to an approximation to the Coulomb potential, can be evaluated as 
\[
\hat{E}_{coul} = \rho^T L^{\dagger}\rho
= [F\rho]^{\ast} D^{\dagger}_g [F \rho].
\]
However, removing these singularities results 
in a constant shift of the total energy, for which a compensation must be made.  
It has been shown in \cite{Ihm} that this compensation can be
calculated by adding a term $E_{\mbox{{\tiny rep}}}$ that 
measures the degree of repulsiveness of the local pseudopotential
with a term that corresponds to the nonsingular part of ion-ion potential 
energy.  Because the second term can be evaluated efficiently by using a 
technique originally developed by Ewald \cite{ewald}, it is denoted by 
$E_{\mbox{{\tiny Ewald}}}$.  Both $E_{\mbox{{\tiny rep}}}$ and 
$E_{\mbox{{\tiny Ewald}}}$ can be computed 
once and for all in a DFT calculation.  We will not go into further details 
of how they are computed since they do not play any role in the algorithms 
we will examine in this section. In KSSOLV, the $E_{\mbox{{\tiny Ewald}}}$ 
calculation is performed in the function {\tt getEwald.m} and 
$E_{\mbox{{\tiny rep}}}$ is calculated in the function {\tt getealpht.m}
Both functions are call once at the beginning of {\tt scf.m}, {\tt dcm.m},
{\tt trdcm.m} etc.

To summarize, the use of a planewave basis allows us to define
a finite-dimensional approximation to the total energy functional 
(\ref{ksetot}) as 
\begin{equation}
\hEtot(X) = \mbox{trace}[X^{\ast}(\frac{1}{2}L + \hat{V}_{ion}) X]
+\frac{1}{2}\rho^T L^{\dagger}\rho + \rho^T \epsilon_{xc}(\rho)
+ E_{\mbox{\tiny Ewald}} + E_{\mbox{\tiny rep}},
\label{fetot}
\end{equation}
where $\hat{V}_{ion}$ denotes the ionic pseudopotentials sampled on 
the suitably chosen Cartesian grid of size $n_1 \times n_2 \times n_3$.

It is easy to verify that the KKT condition associated with the 
constrained minimization problem
\begin{equation}
\min_{X^{\ast}X=I} \hEtot (X)
\label{minetot}
\end{equation}
is
\begin{eqnarray}
H(X) X - X \Lambda_{n_e} &=& 0, \label{kseq} \\
X^{\ast} X &=& I, \nonumber
\end{eqnarray}
where 
\begin{equation}
H(X) = L + \hat{V}_{ion} + \mbox{Diag}(L^{\dagger}\rho) + 
\mbox{Diag}(\mu_{xc}(\rho)),
\label{kshamat}
\end{equation}
$\mu_{xc}(\rho) = d \epsilon_{xc}(\rho) / d \rho$,
and $\Lambda_{n_e}$ is a $n_e \times n_e$ symmetric matrix of 
Lagrangian multipliers.  Because $\hEtot(X) = \hEtot(XQ)$
for any orthogonal matrix $Q \in \mathbb{C}^{n_e \times n_e}$, we
can always choose a particular $Q$ such that $\Lambda_{n_e}$ is
diagonal.  In this case, $\Lambda_{n_e}$ contains $n_e$ eigenvalues
of $H(X)$.  We are interested the $n_e$ smallest eigenvalues and 
the invariant subspace $X$ associated with these eigenvalues.
Because the Hamiltonian matrix $H$ depends on the invariant subspace
to be computed, (\ref{kseq}) is often called a nonlinear 
eigenvalue problem.

\section{SCF}\label{sec:scf}
Currently, the most widely used algorithm for solving
(\ref{kseq}) is the self-consistent field (SCF) iteration which 
we outline in Figure \ref{algscf} for completeness.
%
\begin{figure}[h!]
\begin{center}
\fbox{ \parbox{4.8in}{
\vspace*{.1in}
{SCF Iteration}\\
\\
\begin{tabular}{p{0.5in}p{4.0in}}
{\bf Input}:  & An initial guess of the wavefunction 
                $X^{(0)} \in \mathbb{C}^{n \times n_e}$, 
                pseudopotential; \\
{\bf Output}: & $X \in \mathbb{C}^{n \times n_e}$ such that 
                $X^{\ast} X = I_{n_e}$
                and columns of $X$ spans the invariant subspace
                associated with the smallest $n_e$ eigenvalues of $H(X)$
                defined in (\ref{kshamat}). \\
\\
{\bf \small 1.} & for $i = 1, 2$, ... until convergence\\
{\bf \small 2.} & \hspace*{0.2cm} Form $\Hi = H(X^{(i-1)})$;\\
{\bf \small 3.} & \hspace*{0.2cm} Compute $X^{(i)}$ such that
                  $\Hi X^{(i)} = X^{(i)} \Lambda^{(i)}$, and
                  $\Lambda^{(i)}$ \\
                & \hspace*{0.2cm} contains the $n_e$ smallest eigenvalues
                  of $\Hi$; \\
{\bf \small 4.}& end for \\
\end{tabular} \\
}}
\end{center}
\caption{The SCF iteration} \label{algscf}
\end{figure}

In \cite{trdcm}, we viewed the SCF iteration as an indirect way to minimize
$\hat{E}_{total}$ through the minimization of a sequence of quadratic
surrogate functions of the form
\begin{equation}
q(X) = \frac{1}{2} \mbox{trace}(X^{\ast} \Hi X),
\label{qsur}
\end{equation}
on the manifold $X^{\ast}X=I_{n_e}$.  This constrained minimization problem
is solved in KSSOLV by running a small number of locally optimal
preconditioned conjugate gradient (LOBPCG) iterations \cite{lobpcg}.

Since the surrogate function share the same gradient with $\hat{E}_{total}$ at
$X^{(i)}$, i.e.,
\[
\nabla \hEtot (X)_{|X=X^{(i)}} 
= \Hi X^{(i)} = \nabla q(X)_{|X=X^{(i)}},
\]
moving along a descent direction associated with $q(X)$ is likely to produce a
reduction in $\hat{E}_{total}$.  However, because gradient information
is local, there is no guarantee that the minimizer of $q(X)$, which may 
be far from $\exi$, will yield a lower $\hEtot$ value. This observation
partially explains why SCF often fails to converge. It also suggests
at least two ways to improve the convergence of SCF.

\subsection{Charge mixing}
One possible improvement is to replace the simple gradient-matching 
surrogate $q(X)$ with another quadratic function whose minimizer is 
more likely to yield a reduction in $\hEtot$.  In practice, this alternative 
quadratic function is often constructed by replacing the charge density 
$\rhoi$ in (\ref{kshamat}) with a linear combination of $m$ previously
computed charge densities, i.e.,
\begin{equation}
\rho_{mix} = \sum_{j=0}^{m-1} \alpha_j \rhoin^{(i-j)}
+ \beta_j \rhout^{(i-j)}, \label{rhomix}
\end{equation}
where $\rhoin^{(i-j)}$ is the {\em input} charge density used
to construct $H$ at the $i-j$th iteration, 
$\rhout^{(i-j)}$ is the {\em output} charge density 
obtained from 
\begin{equation}
\rhout^{(i-j)} = f(H(\rhoin^{(i-j)})) \label{frhout}
\end{equation}
and $f(H(\rhoin))$ is a function that we will discuss in the next section.
Here, we can simply think of $f(H(\rhoin^{(i-j)}))$ as the diagonal
part of the projection matrix $XX^{\ast}$, where columns of the $n \times n_e$
matrix $X$ are approximations to the eigenvectors of $H(\rhoin^{(i-j)})$
corresponding to the $n_e$ smallest eigenvalues.

The coefficients $\alpha_0, \alpha_2, ..., \alpha_{i-m+1}$ and
$\beta_0, \beta_2, ..., \beta_{i-m+1}$ in (\ref{rhomix}) are chosen 
to reduce the lack of self consistency between the input charge
density used to define the surrogate function and the output charge density
defined by (\ref{frhout}).  Because $\rho_{mix}$ is a mixture of 
previous input and output change densities, this type of 
surrogate construction technique is often called {\em charge mixing}. 

\subsubsection{Pulay mixing}
A particular charge mixing scheme first proposed by Pulay
for Hartree-Fock calculations \cite{pulay:80,pulay:82} 
chooses the coefficients in (\ref{rhomix}) by solving 
the following constrained minimization problem 
\begin{equation}
\min_{\rho} \| f(\rho) - \rho \|_2^2.
\label{minrho}
\end{equation}
within a subspace that consists of input charge densities computed in previous
SCF iterations.  We will derive Pulay's scheme in this section. 
Note that in computational chemistry, Pulay mixing is referred to as 
the method of {\em direct inversion of iterative subspace} or simply DIIS.  

To derive the Pulay mixing scheme, let us assume that 
$\rhostar$ is the exact solution to the equation
\begin{equation}
\rho = f(\rho). 
\label{fixpt}
\end{equation}
Due to the way $f$ is defined, it is easy to show that
$\rhostar \in  \mathbb{R}^+$ and $\|\rhostar\|_1 = n_e$.

Suppose we already have a few approximations to $\rho$, say, 
$\rho_1, \rho_2, ...,\rho_k$.  Pulay's mixing scheme seeks an optimal 
approximation $\rho_{\mbox{opt}}$ to (\ref{minrho}) from 
\begin{equation}
{\mathcal S} \equiv \{ \rho | \rho = \sum_{i=1}^k\alpha_i\rho_i, \sum_{i=1}^k \alpha_i = 1\}.
\label{subspace}
\end{equation}

It follows form the Taylor's expansion that
\begin{eqnarray}
f(\rhoopt) &=& f(\rhostar) + J_{\ast}(\rhoopt-\rhostar) +
\mbox{higer order terms in } (\rhoopt - \rhostar) \\
           &=& \rhostar + J_{\ast}(\rhoopt-\rhostar) +
\mbox{higer order terms in } (\rhoopt - \rhostar),
\label{taylor}
\end{eqnarray}
where $J_{\ast}$ is the Jacobian matrix of $f$ evaluated at $\rhostar$.

If we ignore the higher order terms in (\ref{taylor}), we obtain
\begin{equation}
f(\rhoopt) - \rhoopt \approx \rhostar- \rhoopt+J_{\ast}(\rhoopt -\rhostar).
\label{approx1st}
\end{equation}

Substituting $\rhoopt = \sum_i \alpha_i \rho_i$ into the right hand side of 
(\ref{approx1st}), we obtain
\begin{eqnarray}
f(\rhoopt) - \rhoopt &\approx& \rhostar - \sum_{i=1}^k\alpha_i \rho_i
-J_{\ast}\biggl[ \rhostar - \sum_{i=1}^k \alpha_i\rho_i \biggr] \nonumber\\
&=& \sum_{i=1}^k\alpha_i (\rhostar - \rho_i) - J_{\ast} 
\biggl[\sum_{i=1}^k \alpha_i (\rhostar-\rho_i) \biggr] \nonumber \\
&=& \sum_{i=1}^k \alpha_i \biggl[
(\rhostar - \rho_i) - J_{\ast} (\rhostar - \rho_i)
\biggr].
\label{newexpand}
\end{eqnarray}

It follow from the Taylor expansion of $f(\rho_i)$ at $\rhostar$ that
\begin{equation}
\Delta \rho_i \equiv f(\rho_i) - \rho_i 
\approx \rhostar - \rho_i - J_{\ast}(\rhostar - \rho_i).
\label{deltarho}
\end{equation}

It now follows from (\ref{newexpand}) and (\ref{deltarho}) that
\begin{equation}
f(\rhoopt) - \rhoopt \approx \sum_{i=1}^k \alpha_i \Delta \rho_i. \label{mixing}
\end{equation}

Therefore, with linear approximation, the best solution to (\ref{minrho})
from the subspace ${\mathcal S}$
can be obtained by solving the following equality constrained quadratic
program
\begin{equation}
\begin{array}{cc}
\min_{\alpha_i} & \| \sum_{i=1}^k \alpha_i \Delta \rho_i \|_2^2 \\
\mbox{s.t.}     &  \sum_{i=1}^k \alpha_i = 1
\end{array}
\label{pulaymin}
\end{equation}

The solution to (\ref{pulaymin}), which can be obtained by eliminating the
constraint and solving a standard linear least squares problem, gives the coefficients used in Pulay mixing.

\subsubsection{Broyden mixing}
Another class of charge mixing algorithms can be derived by 
formulating the Kohn-Sham DFT problem as a system of nonlinear 
equations that involves only the unknown electron charge density $\rho$ and 
the chemical potential $\mu$. These equations can be 
written as
\begin{equation}
F(\rho,\mu) = 
\left(
\begin{array}{c}
r(\rho,\mu) \\
\nu(\rho,\mu)
\end{array}
\right)
=
\left(
\begin{array}{c}
\rho - \mbox{diag}[f_{\beta,\mu}(H(\rho))] \\
\mbox{trace}[f_{\beta,\mu}(H(\rho))] - n_e
\end{array}
\right) = 0,
\label{fdeq}
\end{equation}
where $f_{\beta,\mu}(t)$ is the Fermi-Dirac distribution function
\[
  f_{\beta,\mu,}(t) = \frac{1}{1 + e^{\beta(t-\mu)}},
\]
$\beta$ is a fixed parameter proportional to the reciprocal of the temperature, $H$ is the Hamiltonian of the system, and $n_e$ is the number of electrons.

In theory, we can solve (\ref{fdeq}) by using a standard Newton method, 
which constructs approximations to $\rho$ and $\mu$ iteratively by
\begin{equation}
\left(
\begin{array}{c}
\rho^{(k+1)} \\
\mu^{(k+1)}
\end{array}
\right)
=
\left(
\begin{array}{c}
\rho^{(k)} \\
\mu^{(k)}
\end{array}
\right)
-\hat{J}_k^{-1}
\left(
\begin{array}{c}
r_k \\
\nu_k
\end{array}
\right),
\label{newtonupd}
\end{equation}
where $\hat{J}_k$ is the Jacobian matrix of $F(\rho,\mu)$ evaluated at
$(\rho^{(k)},\mu^{(k)})$, $r_k\equiv r(\rho^{(k)},\mu^{(k)})$ and
$\nu_k \equiv \nu(\rho^{(k)},\mu^{(k)})$.

Because $J_k$ and its inverse are difficult to obtain in practice, 
quasi-Newton methods that replace $J_k$ with an easy-to-evaluate 
approximation are more appropriate.

Note that once $\rhok$ is available, it is easy to find 
$\muk$ such that the second equation in (\ref{fdeq}) is 
satisfied, due to the monotonicity of $\nu(\rho,\mu)$ with respect to $\mu$.
Therefore, we can effectively focus on the first equation in
(\ref{newtonupd}), i.e.,
\begin{equation}
\rho^{(k+1)} = \rho^{(k)} - \tau J_k^{-1} r_k,
\label{rhoupd}
\end{equation}
where $J_k^{-1}$ is the (1,1) block of $\hat{J}_k^{-1}$ and
$\tau$ is an appropriate step length.

Different approximations to $J_k$ or $J_k^{-1}$ yield
different quasi-Newton algorithms. These approximations can be viewed as different
{\em charge mixing} schemes for accelerating the self-consistent 
field (SCF) iteration, and are frequently discussed in the physics 
literature \cite{eyert,kress:cms}.  For example, 
when $J_k^{-1} = I$, (\ref{rhoupd}) reduces to the simple mixing scheme,
\[
\rho^{(k+1)}=(1-\tau)\rho^{(k)} + \tau \mbox{diag}[f_{\beta,\mu}(H(\rho^{(k)}))],
\]
used in the SCF iteration.  

More sophisticated quasi-Newton updating schemes can be devised by using 
Broyden techniques \cite{broyden} to construct better approximations to 
$J_k$ or $J_k^{-1}$. Such techniques yield 
iterative procedures in which an approximation to $J_k$ or $J_k^{-1}$ is 
obtained by performing a sequence of low-rank modifications to some initial 
approximation using a recursive formula \cite{fangsaad,luke}. 

\begin{enumerate}
\item For example, we can construct an approximation to $J_k^{-1}$ by solving 
the following constrained minimization problem
\begin{align}
\min_{C}  \hspace{.5in} &\frac{1}{2}||C - C_{k-1}||^2_F \nonumber \\
\text{s.t } \hspace{.5in} & S_k = CY_k, \label{bmin2}
\end{align}
where $C_{k-1}$ is the approximation to $J^{-1}_{k-1}$, $S_k$ and 
$Y_k$ are defined as 
\begin{eqnarray}
S^{(k)} &=& ( s^{(k)}\: s^{(k-1)} \: \cdots \: s^{(k-\ell)} ), \label{Sk}\\
Y^{(k)} &=& ( y^{(k)}\: y^{(k-1)} \: \cdots \: y^{(k-\ell)} ), \label{Yk}
\end{eqnarray}
and
\begin{equation}
s^{(k)} = \rho^{(k)} - \rho^{(k-1)}, \ \ 
y^{(k)} = r(\rho^{(k)},\mu^{(k)}) - r(\rho^{(k-1)},\mu^{(k)}). 
\label{syk}
\end{equation}
The solution to (\ref{bmin2}) is
\begin{equation}
C_k = C_{k-1} + (S_k-C_{k-1}Y_k)Y_k^{\dagger},
\label{bupd2}
\end{equation}
where $Y_k^{\dagger} = (Y_k^TY_k)^{-1}Y_k^T$ is the pseudoinverse of $Y_k$.

Substituting (\ref{bupd2}) for $J_{k}^{-1}$ in (\ref{rhoupd}) 
yields the following recursive formula for updating $\rhok$:
\begin{equation}
\rho^{(k+1)} = \rhok - \tau C_{k-1}r_k - \tau 
(S_k - C_{k-1}Y_k)Y_k^{\dagger}r_k,
\label{broy2}
\end{equation}
It is easy to show that when $\tau = 1$, (\ref{broy2}) is equivalent to
\[
\rho^{(k+1)} = \rhok
- C_0 r_k - \sum_{j = 1}^k Y_jY_j^{\dagger} r_k + \sum_{j=1}^k S_j Y_j^{\dagger} r_k.
\]

When $\ell=0$, (\ref{broy2}) reduces to the standard Broyden's method of the 
second type. 


\item
We can also construct an approximation to $J_{k-1}$ directly by solving 
\begin{align}
\min_{B}  \hspace{.5in} &\frac{1}{2}||B - B_{k-1}||^2_F \nonumber \\
\text{s.t } \hspace{.5in} & BS_k = Y_k, \label{bmin1}
\end{align}
where $B_{k-1}$ is the approximation to $J_{k-1}$, $S_k$ and $Y_k$
are the same as those defined in (\ref{Sk}) and (\ref{Yk}) respectively.  
The solution to (\ref{bmin1}) yields the following updating formula for $B_k$, 
i.e.,
\begin{equation}
B_k  = B_{k-1} - B_{k-1}S_k S_k^{\dagger} + Y_k S^{\dagger}.
\end{equation}

In order to obtain an approximation to $J_k^{-1}$, which is required in
(\ref{rhoupd}), we invoke the Sherman-Morrison-Woodbury formula that yields
\[
C_k = C_{k-1} + (S_k - C_{k-1}Y_k)(S_k^TC_{k-1}Y_k)^{-1}S_k^TC_{k-1}.
\]
\end{enumerate}

A number of practical issues must be addressed in the implementation
of Broyden schemes.  These issues include
\begin{itemize}
\item The choice of the initial guess to $J$ or $J^{-1}$. A commonly
used initial guess is $\sigma I$ for some $0 < \sigma < 1$. However,
it is not clear how one should select an optimal $\sigma$.
\item The determination of an optimal rank for the Jacobian update. 
      The standard Broyden method performs a rank-1 update of 
      $J_k$ or $J_k^{-1}$ at each iteration. Our preliminary experiments 
      indicate that a multiple rank update \cite{fangsaad} is often more
      effective. However, it is not yet clear what the optimal rank
      should be.
\item The choice of line search parameter $\tau$ in the Broyden
update $\rho^{(k+1)} = \rhok - \gamma J_k^{-1} r_k$. 
\item The choice of when to restart Broyden schemes to minimize storage requirements. 
\end{itemize}

\subsubsection{Anderson mixing}
A special type of mixing scheme called {\em Anderson} mixing \cite{anderson} 
can be derived from (\ref{broy2}) by simply replacing $C_{k-1}$ with
$\sigma I$ and setting $\tau=1$. This yields the much simpler updating formula
\begin{eqnarray*}
\rho^{(k+1)} &=& \rhok - \sigma r_k - (S_k - \sigma Y_k)Y_k^{\dagger}r_k, \\
&=& (\rhok - S_k g) - \sigma (r_k - Y_k g),
\end{eqnarray*}
where $g = Y_k^{\dagger}r_k$.

%We would like to investigate these practical issues and analyze the convergence 
%of Broyden-like iterative methods in terms of the spectral properties
%of the Jacobian near the solution of \eqref{fdeq} and the choice of
%various parameters discussed above.
%
%Another interesting question we would like to address is the accuracy 
%required to evaluate the function
%\begin{equation}
%\mbox{diag}[f_{\beta,\mu}(H(\rho))], \label{fd}
%\end{equation}
%and its impact on the convergence rate of Broyden's method.  Our numerical experiments indicate 
%that existing Broyden's methods often converge even 
%when a large error is made in the function evaluation \eqref{fd} in 
%the first few nonlinear iterations.  In many existing codes, (\ref{fd}) is often 
%evaluated by computing the eigenvalues and eigenvectors of $H(\rho)$ using 
%either a Lanczos algorithm \cite{lanczos} or some type of preconditioned 
%conjugate gradient algorithm \cite{lobpcg}.  The number of iterations used 
%in these algorithms is normally small (5 to 10 iterations) and as a result, 
%the computed eigenvalue and eigenvectors have only a few digits 
%of accuracy at the beginning of the nonlinear iterations. Nevertheless, 
%the Broyden update manages to gradually improve the approximation ($\rhok$) 
%to the solution of \eqref{fdeq}.  As $\rhok$ becomes more accurate,
%the iterative evaluation of (\ref{fd}), which often uses the current
%approximation $\rhok$ as the starting guess, produces more accurate results
%in fewer iterations. A close examination of the connection between
%the accuracy of function evaluation and the convergence rate of the 
%Broyden scheme will allow us to develop effective strategies for
%setting the stopping criterion used in the iterative procedures for
%evaluating \eqref{fd}.  Furthermore, such a study will also allow us to
%determine the accuracy required in other function evaluation methods
%that do not compute the eigenvalue and eigenvectors of $H(\rho)$.
%These methods include the polynomial filtering algorithm proposed in 
%\cite{bekas05,zhou06}, the divide-and-conquer algorithm proposed in 
%\cite{ls3df}, and the diagonal estimator \cite{bekas05} that we will 
%discuss in the next section.  
%
\subsubsection{Other mixing schemes}
Other mixing scheme include {\em Kerker mixing} \cite{kerker},
{\em Thomas-Fermi mixing} \cite{tomfermi}. Charge mixing is often 
quite effective in practice for improving the convergence SCF even though 
its convergence property is still not well understood.  In some cases, 
charge mixing may fail also.

\subsubsection{Choosing a charge mixing scheme in KSSOLV}

KSSOLV supports a number of charge mixing schemes. The default choice
is the Anderson mixing scheme originally developed in \cite{anderson}
and further analyzed in \cite{eyert,fangsaad}.  To change the default choice of 
charge mixing, one must first construct an option structure by
using
\begin{verbatim}
option = setksopt;
\end{verbatim}
This option structure can be modified by using, for example,
\begin{verbatim}
option = setksopt('mixtype','broyden');
\end{verbatim}
which replace the default charge mixing scheme by a Broyden mixing
scheme \cite{luke}.


\subsection{Trust Region}
Another way to improve the convergence of the SCF iteration is to impose an additional constraint to
the surrogate minimization problem (\ref{qsur}) so that the wavefunction
update can be restricted within a small neighborhood of the gradient matching
point $\exi$, thereby ensuring a reduction of the total energy function as we 
minimize the surrogate function.
In \cite{trdcm}, we showed that the following constraint
\[
\| XX^{\ast} - \exi {\exi}^{\ast} \|_F^2 \leq \Delta
\]
is preferred because it is rotationally invariant (i.e., post-multiplying $X$
by an unitary matrix does not change the constraint), and because adding
such a constraint does not increase the complexity of solving the surrogate 
minimization problem.  It is not difficult to show \cite{trdcm} that that solving the
following constrained minimization 
\begin{equation}
\begin{array}{c}
\min q(X) \\
XX^{\ast} = I \\
\| XX^{\ast} - \exi {\exi}^{\ast} \|_F^2 \leq \Delta
\end{array}
\label{trsub}
\end{equation}
is equivalent to solving a low rank perturbed linear eigenvalue problem
\begin{equation}
\biggl[H(\exi) - \sigma \exi {\exi}^{\ast}\biggr] X = X \Lambda, 
\label{shiftev}
\end{equation}
where $\sigma$ is essentially the Lagrange multiplier for the inequality constraint
in (\ref{trsub}) and $\Lambda$ is a diagonal matrix that contains the $n_e$
smallest eigenvalues of the low rank perturbed $\Hi$.
When $\sigma$ is sufficiently large (which corresponds to a trust region
radius $\Delta$ that is sufficiently small), the solution to (\ref{shiftev})
is guaranteed to produce a reduction in $\hEtot(X)$.

\subsection{LOBPCG for Solving Linear Eigenvalue Problems}
As we mentioned earlier, Step 3 of the SCF iteration in which 
one must compute approximations to desired eigenvectors of a fixed Hamiltonian 
must is solved by the LOBPCG algorithm \cite{lobpcg}.  The algorithm
is an iterative procedure that seeks the minimum of the quadratic
surrogate function (\ref{qsur}) by projecting the function into 
a sequence of low dimensional subspaces. The minimizer within each 
subspace, which can be obtained by solving a small generalized eigenvalue
problem produces the optimal search direction and ``step length''
simultaneously.  We will refer users to \cite{lobpcg} for details of
the algorithm.  We would like to point out that a user can easily 
experiment with this algorithm in KSSOLV after a {\tt Ham} object {\tt H} has
been properly constructed. To invoke LOBPCG, one can use
\begin{verbatim} 
[X,lambda,LVEC,RVEC] = lobpcg(H, X0, prec, tol, maxit),
\end{verbatim}
where {\tt H} is a predefined {\tt Ham} object, {\tt X0} is a
{\tt Wavefun} object that contains the initial guess to the 
minimizer of (\ref{qsur}), {\tt prec} is a preconditioner (which
should be constructed as a {\tt Wavefun} object), {\tt tol} is
the convergence tolerance a user must set, and {\tt maxit} is the
maximum number of LOBPCG iterations allowed. Upon completion, 
{\tt lobpcg} returns the approximation to the desired wavefunction 
{\tt X}, the corresponding eigenvalue approximation stored in 
the array {\tt lambda}, and residual norms ({\tt RVEC}) and 
eigenvalues approximation ({\tt LVEC}) computed at each iteration. 

\section{DCM}\label{sec:dcm}
Instead of focusing on Kohn-Sham equations (\ref{kseq}) and minimizing
the total energy indirectly in the SCF iteration, we can minimize the 
total energy directly in an iterative procedure that involves
finding a sequence of search directions along which $\hat{E}_{total}(X)$
decreases and computing an appropriate step length. In most of the earlier 
direct minimization methods developed in 
\cite{arias:92,gillan:89,kress:cms,payne:92,teter:89,vh:2003,voorhis:2002}, 
the search direction and step length computations are carried out separately.
This separation sometimes results in slow convergence.  
We recently developed a new direct constrained minimization (DCM) 
algorithm \cite{dcm,trdcm} in which the search direction and
step length are obtained simultaneously in each iteration by 
minimizing the total energy within a subspace spanned by columns of 
\[
Y = \biggl( X^{(i)}, M^{-1}R^{(i)}, P^{(i-1)} \biggr),
\]
where $X^{(i)}$ is the approximation to $X$ obtained at the $i$th iteration,
$\ri = \Hi \exi - \exi \Lambda^{(i)}$, $M$ is a hermitian positive definite
preconditioner, and $\pim$ is the search direction
obtained in the previous iteration.  It was shown in \cite{dcm} that solving
the subspace minimization problem is equivalent to computing the 
eigenvectors $G$ associated with the $n_e$ smallest eigenvalues of the following
nonlinear eigenvalue problem
\begin{equation}
\hat{H}(G)G = BG\Omega, \ \ G^{\ast}BG = I,
\label{projkseq}
\end{equation}
where 
\begin{equation}
\hat{H}(G) = Y^{\ast} \biggl[ \frac{1}{2}L + V_{ion}
+ \mbox{Diag}\biggl(L^{\dagger} \rho(YG)\biggr)
+ \mbox{Diag}\biggl(\mu_{xc}(\rho(YG))\biggr) \biggr] Y,
\label{projham}
\end{equation}
and $B = Y^{\ast}Y$.

Because the dimension of $\hat{H}(G)$ is at most $3n_e \times 3n_e$, 
which is normally much smaller than that of $H(X)$, it is relatively
easy to solve (\ref{projkseq}) by, for example, a trust region
enabled SCF (TRSCF) iteration.
We should note that it is not necessary to solve (\ref{projkseq})
to full accuracy in the early stage of the DCM algorithm because 
all we need is a $G$ that yields sufficient reduction in the objective
function. 

Once $G$ is obtained, we can update the wave function by
\[
X^{(i+1)} \leftarrow Y G.
\]
The search direction associated with this update is defined,
using the MATLAB submatrix notation, to be
\[
P^{(i)} \equiv Y(:,n_e+1:3n_e)G(n_e+1:3n_e,:).
\]
A complete description of the constrained minimization algorithm
is shown in Figure \ref{alg2}.  We should point out that solving the
projected optimization problem in Step 7 of the algorithm
requires us to evaluate the projected Hamiltonian $\hat{H}(G)$ 
repeatedly as we search for the best $G$.  However, since the first
two terms of $\hat{H}$ do not depend on $G$, they can be
computed and stored in advance. Only the last two terms of
(\ref{projham}) need to be updated.  These updates require
the charge density, the Coulomb and the exchange-correlation
potentials to be recomputed.
%
\begin{figure}[h!]
\begin{center}
\fbox{ \parbox{4.8in}{
\vspace*{.1in}
{{\bf Algorithm}: A Constrained Minimization Algorithm for Total Energy Minimization}\\
\\
\begin{tabular}{p{0.3in}p{4.0in}} 
{\bf Input}:  & initial set of wave functions $\exz \in {\mathbb C}^{n \times n_e}$; 
                ionic pseudopotential; a preconditioner $M$; \\
{\bf Output}: & $X \in \mathbb{C}^{n \times k}$ such that the Kohn-Sham total
                energy functional $E_{total}(X)$ is minimized and $X^{\ast} X = I_k$.
\\
{\bf \small 1.} & Orthonormalize $\exz$ such that ${\exz}^{\ast}\exz=I_k$;\\
{\bf \small 2.} & for $i = 0, 1, 2$, ... until convergence\\
{\bf \small 3.} & \hspace*{0.3cm} Compute $\Theta = {\exi}^{\ast} \Hi \exi$;\\
{\bf \small 4.} & \hspace*{0.3cm} Compute $R = \Hi \exi - \exi \Theta$,\\
{\bf \small 5.} & \hspace*{0.3cm} if ($i > 1$) then \\
                & \hspace*{0.6cm} $Y \leftarrow (\exi, M^{-1}R, P^{(i-1)})$ \\
                & \hspace*{0.3cm} else \\
                & \hspace*{0.6cm} $Y \leftarrow (\exi, M^{-1}R)$; \\
                & \hspace*{0.3cm} endif \\
{\bf \small 6.} & \hspace*{0.3cm} $B \leftarrow Y^{\ast} Y$; \\
{\bf \small 7.} & \hspace*{0.3cm} Find $G \in \mathbb{C}^{2n_e\times n_e} \ \
                  \mbox{or} \ \ \mathbb{C}^{3n_e\times n_e}$ that
                  minimizes $E_{total}(Y G)$ \\
                & \hspace*{0.3cm} subject to the constraint $G^{\ast} B G = I_{n_e}$;\\
{\bf \small 8.} & \hspace*{0.3cm} Set $X^{(i+1)} = Y G$; \\
{\bf \small 9.} & \hspace*{0.3cm} if ($i>1$) then \\
                & \hspace*{0.6cm} $P^{(i)} \leftarrow Y(:,n_e+1:3n_e)G(n_e+1:3n_e,:);$ \\
                & \hspace*{0.3cm} else \\
                & \hspace*{0.6cm} $P^{(i)} \leftarrow Y(:,n_e+1:2n_e)G(n_e+1:2n_e,:)$;\\
                & \hspace*{0.3cm} endif \\
{\bf \small 10.}& end for \\
\end{tabular} \\
}}
\end{center}
\caption{A Direct Constrained Minimization Algorithm for Total Energy Minimization}
\label{alg2}
\end{figure}

\section{Options for SCF and DCM functions} \label{sec:option}
KSSOLV allows one to change a number of parameters used in either
SCF or DCM functions.  These parameters can be changed by defining
and modifying an option structure that can be created by simply calling
\begin{verbatim}
options = setksopt;
\end{verbatim}
This call creates a structure named {\tt options} that contains
a number of default parameters used in SCF and DCM functions. To
see these parameters, one can simply type {\tt options} at the
MATLAB prompt without a semicolon at the end. 
\begin{verbatim}
options = 

       verbose: 'on'
    maxscfiter: 10
    maxdcmiter: 10
    maxinerscf: 3
     maxcgiter: 10
        scftol: 1.0000e-08
        dcmtol: 1.0000e-08
         cgtol: 1.0000e-06
       mixtype: 'anderson'
        mixdim: 9
       betamix: 0.8000
         brank: 1
            X0: []
          rho0: []
        degree: 10
\end{verbatim}

Each field can be modified by simply resetting the value of the field 
through an assignment statement. For example, to change the maximum
number of SCF iterations allowed, we can simply use
\begin{verbatim}
   opts.maxscfiter = 20.
\end{verbatim}
This parameter can also be modified by calling {\tt setksopt} again with
appropriate argument pairs. For example,
\begin{verbatim}
   opts = setksopt(opts,'maxscfiter',20);
\end{verbatim}
achieves the same effect.  The advantage of calling {\tt setksopt} is
that the function checks the validity of the structure field and its
value.

Table~\ref{tab:opts} lists all options that can be defined by {\tt setksopt}
and explains what they are used for.

\begin{table}[htbp]
\center
\begin{tabular}{|c|c|c|} \hline
option name & allowed value & purpose \\ \hline

{\tt verbose} &    'on' or 'off'  &    \begin{minipage}[t]{3in}  
                                       display diagonostic info 
                                       \end{minipage}\\ \hline
{\tt maxscfiter} &  positive integer &    \begin{minipage}[t]{3in}
                                      maximum SCF iterations allowed
                                       \end{minipage}\\ \hline
{\tt maxdcmiter} &  positive integer & \begin{minipage}[t]{3in}
                                      maximum DCM iterations allowed 
                                       \end{minipage}\\ \hline
{\tt maxinerscf} &  positive integer &  \begin{minipage}[t]{3in}
                                       maximum number of inner SCF 
                                       iterations allowed in the DCM 
                                       algorithm
                                       \end{minipage}\\ \hline
{\tt maxcgiter} & positive integer  &   \begin{minipage}[t]{3in}
                                      maximum LOBPCG iterations allowed
                                       \end{minipage}\\ \hline
{\tt scftol} &  positive float &    \begin{minipage}[t]{3in}
                                     convergence tolerance for SCF
                                       \end{minipage}\\ \hline
{\tt dcmtol} &  positive float &    \begin{minipage}[t]{3in}
                                    convergence tolerance for DCM 
                                       \end{minipage}\\ \hline
{\tt cgtol}  &   positive float &    \begin{minipage}[t]{3in}
                                    convergence tolerance for LOBPCG
                                       \end{minipage}\\ \hline
{\tt mixtype} &  \begin{minipage}[t]{1in}
                   'anderson'    
                  'broyden'
                  'broyden1'
                  'pulay'
                  'kerker'
                  'pulay+kerker'
                \end{minipage}
& \begin{minipage}[t]{3in} 
  charge or potential mixing types
  \end{minipage}\\ \hline
{\tt mixdim} &  positive integer &  \begin{minipage}[t]{3in}
                             maximum number of previous potentials to be mixed
                                    \end{minipage}\\ \hline
{\tt betamix} & positive float   &  \begin{minipage}[t]{3in}
  The damping parameter used in Anderson and Broyden mixing
                                    \end{minipage}\\ \hline
{\tt brank} &  positive integer & \begin{minipage}[t]{3in}
                              The rank of the Broyden update
                                    \end{minipage}\\ \hline
{\tt X0} & Wavefun object & \begin{minipage}[t]{3in}
                         An initial guess of the wavefunctions
                                \end{minipage}\\ \hline
{\tt rho0} & 3-D array  & \begin{minipage}[t]{3in}
                         An initial guess of the charge.
                         To restart an SCF or DCM calculation
                         using previously computed results
                         One should pass in both the previously
                         obtained wavefunction and charge.
                                \end{minipage}\\ \hline
{\tt degree} & positive integer  & \begin{minipage}[t]{3in}
                         The degree of the Chebyshev polynomial
                         used in chebyscf.
                      \end{minipage}\\ \hline
\end{tabular}
\caption{Options that can be defined by calling {\tt setksopt}}
\label{tab:opts}
\end{table}
%
%\chapter{KSSOLV Performance} \label{chap:perf}
%The KSSOLV toolbox includes a number of examples that users can 
%experiment with. Each example represents a particular molecule or
%bulk system. The system is created in a setup file.  Table~\ref{egtab}
%shows the names of all setup files and a brief description for each
%one of them. It also shows the number of occupied states ({\tt nocc})
%which is simply the number of electron pairs for most systems
%(with the exception of the quantum dot example in which electrons are 
%not paired by their spin orientations).
%To create a new system, a user can simply take one of the
%existing setup files and modify the construction of the {\tt Molecule}
%object. A user can also change of the size of the supercell or
%the kinetic energy cutoff of the existing setup file to examine
%changes in the convergence of the numerical method or
%the quality of the computed solution.
%%
%
%Table~\ref{egcompare} shows that running an example shown in Table~\ref{egtab}
%typically takes less than a minute on a Linux workstation, with the exception 
%of of the $Pt_2Ni_6O$ example which took more than 10 minutes to complete
%10 SCF iterations.  The timing results reported in
%the table are obtained on a single 2.2 Ghz AMD Opteron processor.
%The total amount of memory available on the machine is 4 gigabytes (GB). 
%A kinetic energy cutoff of $25$ Ryd is used for most systems.
%For a $10\times 10 \times 10$ ($\mbox{\AA}^3$) cubic supercell, such 
%a cutoff results in a $32\times 32 \times 32$ sampling grid for the 
%wavefunctions.  For the $Pt_2Ni_6O$ bulk system, the use of a larger 
%supercell requires the grid size to be increased to $63\times 34 \times 30$. 
%Moreover, because the number of electrons in this heavy-atom system 
%is relatively large ($nocc=43$), the problem takes more time to solve. 
%
%The same initial guesses to the wavefunctions were used for both the SCF and
%DCM runs.  All runs reported in the table used the default parameters
%in SCF and DCM.  For example, in the case of SCF, the maximum number of LOBPCG 
%iterations for solving each linear eigenvalue was set to $10$. In the case of
%DCM, three inner SCF iterations were performed to obtain an approximate
%solution to (\ref{projkseq}).
%
%The SCF and DCM errors reported in the last two columns of the table are
%the residual errors defined as
%\[
%error = \| H(X)X-X \Lambda \|_F,
%\]
%where $X$ contains the wavefunctions returned from SCF or DCM and 
%$\Lambda = X^{\ast} H(X) X$.
%
%Table~\ref{egcompare} shows that DCM appears to run faster than SCF for
%almost all systems. With the exception of two systems ($CO_2$ and quantum 
%dot), it also produces more accurate results. We should caution
%the reader not to interpret the timing provided in Table~\ref{egcompare} 
%as a comparison between the relative efficiency of SCF and DCM 
%because the implementations of these methods have not been optimized
%in KSSOLV.  The optimal implementation of these methods will depend on a 
%number of factors that we will not address in this paper.
%%
%\begin{table}[htbp]
%\center
%\begin{tabular}{|c|c|c||c|c|} \hline
% system      &  SCF time &  DCM time & SCF error & DCM error \\ \hline
%$C_2H_6$     &    44     &     34    &    4.6e-5 &  1.1e-5  \\
%$CO_2$       &    35     &     30    &    1.9e-3 &  1.1e-4  \\
%$H_2O$       &    17     &     17    &    7.9e-5 &  2.0e-5  \\
%$HNCO$       &    48     &     38    &    7.4e-3 &  6.8e-5  \\
%Quantum dot  &    25     &     18    &    5.0e-3 &  3.7e-1  \\
%$Si_2H_4$    &    34     &     31    &    1.8e-3 &  2.7e-4  \\
%silicon bulk &    15     &     17    &    3.0e-4 &  9.6e-6  \\
%$SiH_4$      &    29     &     24    &    9.7e-6 &  4.9e-7  \\
%$Pt_2Ni_6O$  &   1108    &    480    &    3.7    &  4.9e-2  \\
%\hline
%\end{tabular}
%\caption{Comparing the timing and accuracy of running SCF and DCM in KSSOLV}
%\label{egcompare}
%\end{table}
%
\bibliographystyle{plain}
\bibliography{kssolvug}
\addtocounter{page}{1}
\addcontentsline{toc}{chapter}{Bibliography}
\addtocounter{page}{-1}
\end{document}
